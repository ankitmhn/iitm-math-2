\documentclass{article}
\usepackage{amsmath}
\usepackage{amsfonts}
\usepackage{amssymb}
\usepackage{geometry}

\geometry{margin=1in}

\title{Lecture Summary: Matrices}
\author{}
\date{}

\begin{document}

\maketitle

\section*{Source: Lec 17.pdf}

\section*{Key Points}

\begin{itemize}
  \item \textbf{Definition of Matrices:}
    \begin{itemize}
      \item A matrix is a rectangular array of numbers arranged in rows and columns.
      \item Described as $m \times n$, where $m$ is the number of rows, and $n$ is the number of columns.
      \item The $(i, j)$-th entry is the element at the intersection of the $i$-th row and $j$-th column.
    \end{itemize}

  \item \textbf{Special Types of Matrices:}
    \begin{itemize}
      \item \textbf{Square Matrix:} Number of rows equals the number of columns.
      \item \textbf{Diagonal Matrix:} Non-zero entries only on the diagonal.
      \item \textbf{Scalar Matrix:} A diagonal matrix with all diagonal entries equal.
      \item \textbf{Identity Matrix ($I$):} A scalar matrix with all diagonal entries equal to 1.
    \end{itemize}

  \item \textbf{Operations on Matrices:}
    \begin{itemize}
      \item \textbf{Addition:} Add corresponding entries of two matrices of the same size.
      \item \textbf{Scalar Multiplication:} Multiply each matrix entry by a scalar.
      \item \textbf{Matrix Multiplication:} Multiply rows of the first matrix with columns of the second matrix and sum the products.
        \begin{itemize}
          \item Defined only if the number of columns in the first matrix equals the number of rows in the second matrix.
        \end{itemize}
    \end{itemize}

  \item \textbf{Properties:}
    \begin{itemize}
      \item Matrix addition is commutative ($A + B = B + A$) and associative.
      \item Matrix multiplication is associative but not commutative.
      \item Distributive properties:
        \begin{itemize}
          \item $A(B + C) = AB + AC$
          \item $(A + B)C = AC + BC$
          \item $\lambda(AB) = (\lambda A)B = A(\lambda B)$, where $\lambda$ is a scalar.
        \end{itemize}
    \end{itemize}

  \item \textbf{Application to Linear Equations:}
    \begin{itemize}
      \item Matrices can represent systems of linear equations, making them central to solving such systems.
    \end{itemize}
\end{itemize}

\section*{Simplified Explanation}

A matrix is like a table of numbers with rows and columns. For example:
\[
  \text{A } 2 \times 3 \text{ matrix has 2 rows and 3 columns:}
\]
\[
  \begin{bmatrix}
    1 & 2 & 3 \\
    4 & 5 & 6
  \end{bmatrix}.
\]
The $(1, 2)$-th entry is 2, found in the first row and second column.

Special matrices include:
\begin{itemize}
  \item \textbf{Diagonal Matrix:} All non-diagonal entries are zero, e.g.,
    \[
      \begin{bmatrix}
        1 & 0 & 0 \\
        0 & -3 & 0 \\
        0 & 0 & 4.2
      \end{bmatrix}.
    \]
  \item \textbf{Scalar Matrix:} A diagonal matrix where all diagonal entries are the same, e.g.,
    \[
      \begin{bmatrix}
        -3 & 0 & 0 \\
        0 & -3 & 0 \\
        0 & 0 & -3
      \end{bmatrix}.
    \]
  \item \textbf{Identity Matrix ($I$):} A scalar matrix where diagonal entries are all 1, e.g.,
    \[
      \begin{bmatrix}
        1 & 0 & 0 \\
        0 & 1 & 0 \\
        0 & 0 & 1
      \end{bmatrix}.
    \]
\end{itemize}

\section*{Operations Examples}

\begin{enumerate}
  \item \textbf{Addition:}
    Add corresponding entries of two matrices:
    \[
      \begin{bmatrix}
        1 & 2 \\
        3 & 4
      \end{bmatrix} +
      \begin{bmatrix}
        5 & 6 \\
        7 & 8
      \end{bmatrix} =
      \begin{bmatrix}
        6 & 8 \\
        10 & 12
      \end{bmatrix}.
    \]

  \item \textbf{Scalar Multiplication:}
    Multiply each entry of the matrix by 3:
    \[
      3 \times
      \begin{bmatrix}
        1 & 2 \\
        3 & 4
      \end{bmatrix} =
      \begin{bmatrix}
        3 & 6 \\
        9 & 12
      \end{bmatrix}.
    \]

  \item \textbf{Matrix Multiplication:}
    Multiply a $2 \times 2$ matrix by another $2 \times 2$ matrix:
    \[
      \begin{bmatrix}
        1 & 2 \\
        3 & 4
      \end{bmatrix} \times
      \begin{bmatrix}
        5 & 6 \\
        7 & 8
      \end{bmatrix} =
      \begin{bmatrix}
        19 & 22 \\
        43 & 50
      \end{bmatrix}.
    \]
    Each entry is calculated by multiplying rows with columns and summing the products.
\end{enumerate}

\section*{Practical Importance}

Matrices are foundational in solving systems of linear equations and are widely used in fields like physics, computer science, and data analysis.

By understanding their structure, operations, and properties, you can efficiently represent and solve complex problems. For example, the matrix representation of the system:
\[
  3x + 4y = 5, \quad 4x + 6y = 10
\]
is:
\[
  \begin{bmatrix}
    3 & 4 & 5 \\
    4 & 6 & 10
  \end{bmatrix}.
\]
This forms the basis for advanced methods like Gaussian elimination.

\end{document}
