\documentclass{article}
\usepackage{amsmath}
\usepackage{amssymb}
\usepackage{geometry}

\geometry{margin=1in}

\title{Lecture Summary: Linear Mappings – Part 2}
\author{}
\date{}

\begin{document}

\maketitle

\section*{Source: Lec39.pdf}

\section*{Key Points}

\begin{itemize}
  \item \textbf{Review of Linear Mappings:}
    \begin{itemize}
      \item A linear mapping is a function $f: \mathbb{R}^n \to \mathbb{R}^m$ that satisfies:
        \[
          f(a \cdot x + b \cdot y) = a \cdot f(x) + b \cdot f(y),
        \]
        where $a, b \in \mathbb{R}$ and $x, y \in \mathbb{R}^n$.
      \item Linear mappings preserve the operations of addition and scalar multiplication.
    \end{itemize}

  \item \textbf{Examples: Cost Functions in Multiple Shops:}
    \begin{itemize}
      \item Shop A prices:
        \[
          c_A(x_1, x_2, x_3) = 45x_1 + 125x_2 + 150x_3.
        \]
      \item Shop B prices:
        \[
          c_B(x_1, x_2, x_3) = 40x_1 + 120x_2 + 170x_3.
        \]
      \item Shop C prices:
        \[
          c_C(x_1, x_2, x_3) = 50x_1 + 130x_2 + 160x_3.
        \]
      \item These functions can be written in matrix form:
        \[
          c(x_1, x_2, x_3) =
          \begin{bmatrix}
            45 & 125 & 150 \\
            40 & 120 & 170 \\
            50 & 130 & 160
          \end{bmatrix}
          \begin{bmatrix}
            x_1 \\ x_2 \\ x_3
          \end{bmatrix}.
        \]
    \end{itemize}

  \item \textbf{Using Linear Mappings for Decision-Making:}
    \begin{itemize}
      \item To compare costs across shops, evaluate the cost functions for given quantities of rice, dal, and oil.
      \item Example: For $x_1 = 2$ kg (rice), $x_2 = 1$ kg (dal), $x_3 = 2$ liters (oil):
        \[
          c_A(2, 1, 2) = 515, \quad c_B(2, 1, 2) = 540, \quad c_C(2, 1, 2) = 550.
        \]
      \item Shop A is the most economical in this scenario.
    \end{itemize}

  \item \textbf{General Form of Linear Mappings:}
    \begin{itemize}
      \item A general linear mapping $f: \mathbb{R}^n \to \mathbb{R}^m$ has the form:
        \[
          f(x_1, x_2, \dots, x_n) =
          \begin{bmatrix}
            \sum_{j=1}^n a_{1j}x_j \\
            \sum_{j=1}^n a_{2j}x_j \\
            \vdots \\
            \sum_{j=1}^n a_{mj}x_j
          \end{bmatrix},
        \]
        where $a_{ij} \in \mathbb{R}$ are constants.
      \item This can be expressed as:
        \[
          f(x) = A \cdot x,
        \]
        where $A$ is an $m \times n$ matrix and $x$ is a column vector.
    \end{itemize}

  \item \textbf{Linearity of Mappings:}
    \begin{itemize}
      \item Linearity ensures that:
        \[
          f(x + c \cdot y) = f(x) + c \cdot f(y),
        \]
        where $x, y \in \mathbb{R}^n$ and $c \in \mathbb{R}$.
      \item This property simplifies computations, allowing for decomposition and scaling of inputs.
    \end{itemize}
\end{itemize}

\section*{Simplified Explanation}

\textbf{Example: Cost Vector for Three Shops}
Prices for rice, dal, and oil:
\[
  \text{Shop A: } [45, 125, 150], \quad
  \text{Shop B: } [40, 120, 170], \quad
  \text{Shop C: } [50, 130, 160].
\]
\begin{itemize}
  \item For $x_1 = 2$ (rice), $x_2 = 1$ (dal), $x_3 = 2$ (oil):
    \[
      c_A(2, 1, 2) = 515, \quad c_B(2, 1, 2) = 540, \quad c_C(2, 1, 2) = 550.
    \]
  \item Matrix form:
    \[
      c(x) =
      \begin{bmatrix}
        45 & 125 & 150 \\
        40 & 120 & 170 \\
        50 & 130 & 160
      \end{bmatrix}
      \begin{bmatrix}
        x_1 \\ x_2 \\ x_3
      \end{bmatrix}.
    \]
\end{itemize}

\section*{Conclusion}

In this lecture, we:
\begin{itemize}
  \item Explored linear mappings through real-world cost functions in multiple shops.
  \item Defined the general form of a linear mapping using matrices.
  \item Highlighted the computational advantages of linearity in comparing and analyzing functions.
\end{itemize}

Linear mappings are powerful tools for representing and solving problems involving linear relationships in various dimensions.

\end{document}
