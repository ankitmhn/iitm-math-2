\documentclass{article}
\usepackage{amsmath}
\usepackage{amssymb}
\usepackage{geometry}

\geometry{margin=1in}

\title{Lecture Summary: Partial Derivatives}
\author{}
\date{}

\begin{document}

\maketitle

\section*{Source: Partial Derivatives.pdf}

\section*{Key Points}

\begin{itemize}
  \item \textbf{Definition of Partial Derivatives:}
    \begin{itemize}
      \item For a scalar-valued function $f(x_1, x_2, \dots, x_n)$ defined on a domain $D \subset \mathbb{R}^n$, the partial derivative with respect to $x_i$ at a point $\vec{a} \in \mathbb{R}^n$ is:
        \[
          \frac{\partial f}{\partial x_i}(\vec{a}) = \lim_{h \to 0} \frac{f(\vec{a} + h\vec{e}_i) - f(\vec{a})}{h},
        \]
        where $\vec{e}_i$ is the $i$th standard basis vector.
      \item Partial derivatives measure the rate of change of $f$ with respect to one variable, keeping all other variables constant.
    \end{itemize}

  \item \textbf{Interpretation:}
    \begin{itemize}
      \item Partial derivatives generalize the concept of derivatives from single-variable calculus to multivariable functions.
      \item They capture the slope of the function in the direction of the $i$th coordinate axis.
    \end{itemize}

  \item \textbf{Examples of Partial Derivatives:}
    \begin{itemize}
      \item Example 1: For $f(x, y) = x + y$:
        \[
          \frac{\partial f}{\partial x} = 1, \quad \frac{\partial f}{\partial y} = 1.
        \]
      \item Example 2: For $f(x, y, z) = xy + yz + zx$:
        \[
          \frac{\partial f}{\partial x} = y + z, \quad \frac{\partial f}{\partial y} = x + z, \quad \frac{\partial f}{\partial z} = x + y.
        \]
      \item Example 3: For $f(x, y) = \sin(xy)$:
        \[
          \frac{\partial f}{\partial x} = y\cos(xy), \quad \frac{\partial f}{\partial y} = x\cos(xy).
        \]
    \end{itemize}

  \item \textbf{Computing Partial Derivatives:}
    \begin{itemize}
      \item To compute $\frac{\partial f}{\partial x_i}$:
        \begin{enumerate}
          \item Treat $x_i$ as the only variable.
          \item Treat all other variables as constants.
          \item Differentiate $f$ with respect to $x_i$ using standard differentiation rules.
        \end{enumerate}
      \item For well-behaved (smooth) functions, this process is straightforward. For piecewise or non-smooth functions, limits may be required.
    \end{itemize}

  \item \textbf{Applications of Partial Derivatives:}
    \begin{itemize}
      \item Analyze the behavior of multivariable functions, such as rates of change in physical systems.
      \item Compute gradients and optimize multivariable functions.
      \item Formulate equations in physics, engineering, and economics.
    \end{itemize}
\end{itemize}

\section*{Simplified Explanation}

\textbf{What Are Partial Derivatives?}
Partial derivatives measure how a multivariable function changes when only one variable changes, keeping the others fixed.

\textbf{How to Compute Them?}
Treat one variable as the only variable and differentiate as usual, treating all others as constants.

\textbf{Example:}
For $f(x, y) = x^2 + 3y$:
\[
  \frac{\partial f}{\partial x} = 2x, \quad \frac{\partial f}{\partial y} = 3.
\]

\section*{Conclusion}

In this lecture, we:
\begin{itemize}
  \item Defined partial derivatives as a natural extension of single-variable derivatives.
  \item Explored examples to compute partial derivatives.
  \item Highlighted their applications in analyzing multivariable systems.
\end{itemize}

Partial derivatives are foundational in multivariable calculus, enabling detailed exploration of functions with several variables.

\end{document}
