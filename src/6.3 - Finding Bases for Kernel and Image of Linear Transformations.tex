\documentclass{article}
\usepackage{amsmath}
\usepackage{amssymb}
\usepackage{geometry}

\geometry{margin=1in}

\title{Lecture Summary: Examples of Finding Bases for Kernel and Image of Linear Transformations}
\author{}
\date{}

\begin{document}

\maketitle

\section*{Source: Lec43.pdf}

\section*{Key Points}

\begin{itemize}
  \item \textbf{Review of Kernel and Image:}
    \begin{itemize}
      \item The kernel $\ker(f)$ of a linear transformation $f: V \to W$ is:
        \[
          \ker(f) = \{v \in V \mid f(v) = 0\}.
        \]
      \item The image $\operatorname{Im}(f)$ is:
        \[
          \operatorname{Im}(f) = \{w \in W \mid \exists v \in V, \, w = f(v)\}.
        \]
      \item The kernel corresponds to the null space of the associated matrix, and the image corresponds to the column space.
    \end{itemize}

  \item \textbf{Finding Bases Using Row Reduction:}
    \begin{itemize}
      \item Row reduce the matrix representation of $f$ to obtain:
        \begin{enumerate}
          \item Basis for the null space (kernel) using the non-pivot columns.
          \item Basis for the column space (image) using the pivot columns.
        \end{enumerate}
    \end{itemize}

  \item \textbf{Example 1: Transformation from $\mathbb{R}^4$ to $\mathbb{R}^3$:}
    \begin{itemize}
      \item Transformation defined by:
        \[
          T(x_1, x_2, x_3, x_4) =
          \begin{bmatrix}
            2x_1 + 4x_2 + 6x_3 + 8x_4 \\
            x_1 + 3x_2 + 5x_4 \\
            x_1 + x_2 + 6x_3 + 3x_4
          \end{bmatrix}.
        \]
      \item Matrix representation:
        \[
          A =
          \begin{bmatrix}
            2 & 4 & 6 & 8 \\
            1 & 3 & 0 & 5 \\
            1 & 1 & 6 & 3
          \end{bmatrix}.
        \]
      \item Row reduce $A$:
        \[
          \begin{bmatrix}
            1 & 0 & 9 & 2 \\
            0 & 1 & -3 & 1 \\
            0 & 0 & 0 & 0
          \end{bmatrix}.
        \]
      \item Kernel basis:
        \begin{itemize}
          \item Non-pivot columns: $x_3, x_4$ are independent variables.
          \item Solve for $x_1, x_2$:
            \[
              x_1 = -9x_3 - 2x_4, \quad x_2 = 3x_3 - x_4.
            \]
          \item Basis vectors:
            \[
              (-9, 3, 1, 0), \quad (-2, -1, 0, 1).
            \]
        \end{itemize}
      \item Image basis:
        \begin{itemize}
          \item Pivot columns: 1 and 2.
          \item Basis vectors from original matrix:
            \[
              (2, 1, 1), \quad (4, 3, 1).
            \]
        \end{itemize}
    \end{itemize}

  \item \textbf{Example 2: Transformation from $\mathbb{R}^2$ to $\mathbb{R}^3$:}
    \begin{itemize}
      \item Transformation defined by:
        \[
          T(x, y) =
          \begin{bmatrix}
            0 \\
            x + 2y \\
            -x - 2y
          \end{bmatrix}.
        \]
      \item Basis $\beta = \{(1, 1), (1, -1)\}$ for $\mathbb{R}^2$, $\gamma = \{(-1, 1, 0), (-1, 0, 1)\}$ for $\mathbb{R}^3$.
      \item Compute $T(1, 1)$:
        \[
          T(1, 1) =
          \begin{bmatrix}
            0 \\
            3 \\
            -3
          \end{bmatrix} = 3(-1, 1, 0) + (-3)(-1, 0, 1).
        \]
      \item Compute $T(1, -1)$:
        \[
          T(1, -1) =
          \begin{bmatrix}
            0 \\
            -1 \\
            1
          \end{bmatrix} = -1(-1, 1, 0) + 1(-1, 0, 1).
        \]
      \item Matrix representation:
        \[
          A =
          \begin{bmatrix}
            3 & -1 \\
            -3 & 1
          \end{bmatrix}.
        \]
      \item Row reduce $A$:
        \[
          \begin{bmatrix}
            1 & -\frac{1}{3} \\
            0 & 0
          \end{bmatrix}.
        \]
      \item Kernel basis:
        \[
          \left(\frac{1}{3}, 1\right).
        \]
      \item Image basis:
        \[
          \{(-1, 1, 0)\}.
        \]
    \end{itemize}

  \item \textbf{Rank-Nullity Theorem for Linear Transformations:}
    \begin{itemize}
      \item Rank of $T$ = dimension of image.
      \item Nullity of $T$ = dimension of kernel.
      \item Rank-nullity theorem:
        \[
          \text{rank}(T) + \text{nullity}(T) = \dim(V).
        \]
    \end{itemize}
\end{itemize}

\section*{Simplified Explanation}

\textbf{Example 1: Basis for Kernel and Image}
Matrix:
\[
  A =
  \begin{bmatrix}
    2 & 4 & 6 & 8 \\
    1 & 3 & 0 & 5 \\
    1 & 1 & 6 & 3
  \end{bmatrix}.
\]
\begin{itemize}
  \item Row reduce to:
    \[
      \begin{bmatrix}
        1 & 0 & 9 & 2 \\
        0 & 1 & -3 & 1 \\
        0 & 0 & 0 & 0
      \end{bmatrix}.
    \]
  \item Kernel basis:
    \[
      (-9, 3, 1, 0), \quad (-2, -1, 0, 1).
    \]
  \item Image basis:
    \[
      (2, 1, 1), \quad (4, 3, 1).
    \]
\end{itemize}

\textbf{Example 2: Custom Basis}
Matrix:
\[
  A =
  \begin{bmatrix}
    3 & -1 \\
    -3 & 1
  \end{bmatrix}.
\]
\begin{itemize}
  \item Row reduce to:
    \[
      \begin{bmatrix}
        1 & -\frac{1}{3} \\
        0 & 0
      \end{bmatrix}.
    \]
  \item Kernel basis:
    \[
      \left(\frac{1}{3}, 1\right).
    \]
  \item Image basis:
    \[
      \{(-1, 1, 0)\}.
    \]
\end{itemize}

\section*{Conclusion}

In this lecture, we:
\begin{itemize}
  \item Demonstrated the computation of kernel and image bases for linear transformations.
  \item Illustrated the use of row reduction to find these bases.
  \item Revisited the rank-nullity theorem in the context of linear transformations.
\end{itemize}

These examples reinforce the practical application of kernel and image concepts in linear algebra.

\end{document}
