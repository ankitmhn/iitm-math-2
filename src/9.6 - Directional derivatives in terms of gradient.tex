\documentclass{article}
\usepackage{amsmath}
\usepackage{amssymb}
\usepackage{geometry}

\geometry{margin=1in}

\title{Lecture Summary: Directional Derivatives in Terms of the Gradient}
\author{}
\date{}

\begin{document}

\maketitle

\section*{Source: Directional derivatives in terms of the gradient.pdf}

\section*{Key Points}

\begin{itemize}
  \item \textbf{Definition of Gradient:}
    \begin{itemize}
      \item The gradient of a scalar function $f(x_1, x_2, \dots, x_n)$ is:
        \[
          \nabla f = \left( \frac{\partial f}{\partial x_1}, \frac{\partial f}{\partial x_2}, \dots, \frac{\partial f}{\partial x_n} \right).
        \]
      \item It is a vector-valued function from $\mathbb{R}^n$ to $\mathbb{R}^n$.
      \item At a point $\vec{a} \in \mathbb{R}^n$, the gradient vector is $\nabla f(\vec{a})$.
    \end{itemize}

  \item \textbf{Directional Derivatives in Terms of the Gradient:}
    \begin{itemize}
      \item The directional derivative of $f$ at $\vec{a}$ in the direction of a unit vector $\vec{u}$ is:
        \[
          D_{\vec{u}} f(\vec{a}) = \nabla f(\vec{a}) \cdot \vec{u}.
        \]
      \item This simplifies the computation of directional derivatives compared to using limits directly.
    \end{itemize}

  \item \textbf{Examples:}
    \begin{itemize}
      \item Example 1: For $f(x, y) = x + y$, the gradient is $\nabla f = (1, 1)$.
        \[
          D_{\vec{u}} f(x, y) = (1, 1) \cdot \vec{u} = u_1 + u_2.
        \]
      \item Example 2: For $f(x, y, z) = xy + yz + zx$, the gradient is $\nabla f = (y + z, x + z, x + y)$.
        \[
          D_{\vec{u}} f(x, y, z) = (y + z, x + z, x + y) \cdot \vec{u} = u_1(y + z) + u_2(x + z) + u_3(x + y).
        \]
      \item Example 3: For $f(x, y) = \sin(xy)$, the gradient is $\nabla f = (y\cos(xy), x\cos(xy))$.
        \[
          D_{\vec{u}} f(x, y) = (y\cos(xy), x\cos(xy)) \cdot \vec{u} = u_1 y \cos(xy) + u_2 x \cos(xy).
        \]
    \end{itemize}

  \item \textbf{Properties of the Gradient:}
    \begin{itemize}
      \item \textbf{Linearity:}
        \[
          \nabla (cf + g) = c \nabla f + \nabla g.
        \]
      \item \textbf{Product Rule:}
        \[
          \nabla (fg) = g \nabla f + f \nabla g.
        \]
      \item \textbf{Quotient Rule:}
        \[
          \nabla \left(\frac{f}{g}\right) = \frac{g \nabla f - f \nabla g}{g^2}.
        \]
    \end{itemize}

  \item \textbf{Importance of the Gradient:}
    \begin{itemize}
      \item The gradient vector $\nabla f$ at $\vec{a}$ indicates the direction of steepest ascent of $f$ at that point.
      \item The magnitude $\|\nabla f(\vec{a})\|$ gives the rate of steepest ascent.
    \end{itemize}

  \item \textbf{Continuity of the Gradient:}
    \begin{itemize}
      \item If $\nabla f$ is continuous around $\vec{a}$, then $D_{\vec{u}} f(\vec{a})$ exists and equals $\nabla f(\vec{a}) \cdot \vec{u}$.
      \item If $\nabla f$ is not continuous, the formula $D_{\vec{u}} f(\vec{a}) = \nabla f(\vec{a}) \cdot \vec{u}$ may fail.
    \end{itemize}
\end{itemize}

\section*{Simplified Explanation}

\textbf{Gradient Vector:}
A vector of partial derivatives that points in the direction of steepest ascent.

\textbf{Directional Derivative:}
Instead of using limits, compute the directional derivative using:
\[
  D_{\vec{u}} f(\vec{a}) = \nabla f(\vec{a}) \cdot \vec{u}.
\]

\textbf{Example:}
For $f(x, y) = x + y$ and $\vec{u} = (1/\sqrt{2}, 1/\sqrt{2})$:
\[
  D_{\vec{u}} f(0, 0) = (1, 1) \cdot (1/\sqrt{2}, 1/\sqrt{2}) = \sqrt{2}.
\]

\section*{Conclusion}

In this lecture, we:
\begin{itemize}
  \item Defined the gradient and its role in computing directional derivatives.
  \item Highlighted the connection between gradients and rates of change.
  \item Demonstrated examples to illustrate the use of gradients in multivariable calculus.
\end{itemize}

The gradient simplifies directional derivative calculations and provides geometric insights into the behavior of multivariable functions.

\end{document}
