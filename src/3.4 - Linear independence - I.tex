\documentclass{article}
\usepackage{amsmath}
\usepackage{amssymb}
\usepackage{geometry}

\geometry{margin=1in}

\title{Lecture Summary: Linear Independence – Part 1}
\author{}
\date{}

\begin{document}

\maketitle

\section*{Source: Lec30.pdf}

\section*{Key Points}

\begin{itemize}
  \item \textbf{Definition of Linear Independence:}
    \begin{itemize}
      \item A set of vectors $v_1, v_2, \dots, v_n$ from a vector space $V$ is \textbf{linearly independent} if:
        \[
          a_1 v_1 + a_2 v_2 + \dots + a_n v_n = 0 \implies a_1 = a_2 = \dots = a_n = 0.
        \]
      \item Linear independence means the only way to express the zero vector as a linear combination of $v_1, v_2, \dots, v_n$ is with all coefficients $a_i = 0$.
    \end{itemize}

  \item \textbf{Relation to Linear Dependence:}
    \begin{itemize}
      \item A set of vectors is linearly dependent if there exist scalars $a_1, a_2, \dots, a_n$, not all zero, such that:
        \[
          a_1 v_1 + a_2 v_2 + \dots + a_n v_n = 0.
        \]
      \item Linear independence is the negation of linear dependence.
    \end{itemize}

  \item \textbf{Key Properties:}
    \begin{itemize}
      \item Any set of vectors containing the zero vector is linearly dependent.
      \item Two non-zero vectors are linearly independent if and only if they are not scalar multiples of each other.
      \item Three vectors in $\mathbb{R}^3$ are linearly independent if no vector is a linear combination of the other two.
    \end{itemize}

  \item \textbf{Examples:}
    \begin{itemize}
      \item In $\mathbb{R}^2$, vectors $(-1, 3)$ and $(2, 0)$ are linearly independent:
        \[
          a(-1, 3) + b(2, 0) = (0, 0) \implies a = 0, b = 0.
        \]
      \item A set containing the zero vector, such as $\{(0, 0), (1, 2)\}$, is linearly dependent.
      \item In $\mathbb{R}^3$, vectors $(1, 1, 2)$, $(1, 2, 0)$, and $(0, 2, 1)$ are linearly independent:
        \[
          a(1, 1, 2) + b(1, 2, 0) + c(0, 2, 1) = (0, 0, 0) \implies a = 0, b = 0, c = 0.
        \]
    \end{itemize}

  \item \textbf{Geometric Interpretation:}
    \begin{itemize}
      \item Two vectors are linearly independent if they point in different directions (not collinear).
      \item Three vectors in $\mathbb{R}^3$ are linearly independent if they do not lie on the same plane.
    \end{itemize}
\end{itemize}

\section*{Simplified Explanation}

\textbf{Example 1: Linear Independence in $\mathbb{R}^2$}
Vectors $(-1, 3)$ and $(2, 0)$ satisfy:
\[
  a(-1, 3) + b(2, 0) = (0, 0) \implies -a + 2b = 0, \quad 3a = 0.
\]
Solving gives $a = 0$, $b = 0$, so the vectors are linearly independent.

\textbf{Example 2: Dependency with the Zero Vector}
If $v_1 = (0, 0)$ and $v_2 = (1, 2)$:
\[
  a(0, 0) + b(1, 2) = (0, 0) \implies b = 0, \quad a \text{ can be non-zero}.
\]
The set is linearly dependent since the zero vector is included.

\textbf{Example 3: Linear Independence in $\mathbb{R}^3$}
Vectors $(1, 1, 2)$, $(1, 2, 0)$, and $(0, 2, 1)$:
\[
  a(1, 1, 2) + b(1, 2, 0) + c(0, 2, 1) = (0, 0, 0).
\]
Solving gives $a = 0$, $b = 0$, $c = 0$, so the vectors are linearly independent.

\section*{Conclusion}

In this lecture, we:
\begin{itemize}
  \item Defined linear independence and its relationship with linear dependence.
  \item Presented key examples in $\mathbb{R}^2$ and $\mathbb{R}^3$ to illustrate the concept.
  \item Highlighted geometric interpretations and practical scenarios.
\end{itemize}

Understanding linear independence is essential for analyzing vector spaces, solving linear systems, and exploring higher-dimensional geometry.

\end{document}
