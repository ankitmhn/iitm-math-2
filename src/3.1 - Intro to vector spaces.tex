\documentclass{article}
\usepackage{amsmath}
\usepackage{amssymb}
\usepackage{geometry}

\geometry{margin=1in}

\title{Lecture Summary: Introduction to Vector Spaces}
\author{}
\date{}

\begin{document}

\maketitle

\section*{Source: Lec27.pdf}

\section*{Key Points}

\begin{itemize}
  \item \textbf{Definition of Vector Spaces:}
    \begin{itemize}
      \item A vector space is a set $V$ with two operations:
        \begin{enumerate}
          \item Vector addition: $+ : V \times V \to V$.
          \item Scalar multiplication: $\cdot : \mathbb{R} \times V \to V$.
        \end{enumerate}
      \item These operations must satisfy 8 axioms, abstracted from $\mathbb{R}^n$.
    \end{itemize}

  \item \textbf{Axioms of Vector Spaces:}
    For all $v, w, u \in V$ and scalars $a, b \in \mathbb{R}$:
    \begin{enumerate}
      \item $v + w = w + v$ (commutativity).
      \item $(v + w) + u = v + (w + u)$ (associativity of addition).
      \item There exists a zero vector $0 \in V$ such that $v + 0 = v$.
      \item For every $v \in V$, there exists $-v \in V$ such that $v + (-v) = 0$.
      \item $1 \cdot v = v$, where $1$ is the multiplicative identity in $\mathbb{R}$.
      \item $a \cdot (b \cdot v) = (a \cdot b) \cdot v$ (associativity of scalar multiplication).
      \item $a \cdot (v + w) = a \cdot v + a \cdot w$ (distributivity of scalar multiplication over vector addition).
      \item $(a + b) \cdot v = a \cdot v + b \cdot v$ (distributivity of scalar addition over scalar multiplication).
    \end{enumerate}

  \item \textbf{Examples of Vector Spaces:}
    \begin{itemize}
      \item $\mathbb{R}^n$ with standard addition and scalar multiplication.
      \item The set of $m \times n$ matrices with real entries, where:
        \begin{enumerate}
          \item Matrix addition is element-wise.
          \item Scalar multiplication is element-wise scaling.
        \end{enumerate}
      \item The solution set of a homogeneous system $Ax = 0$ forms a vector space.
    \end{itemize}

  \item \textbf{Non-Examples of Vector Spaces:}
    \begin{itemize}
      \item Redefining addition or scalar multiplication in $\mathbb{R}^2$ can lead to violations of the axioms. For example:
        \begin{enumerate}
          \item Redefining addition such that the second component subtracts instead of adding can break commutativity.
          \item Redefining scalar multiplication to ignore one component can break distributivity.
        \end{enumerate}
    \end{itemize}

  \item \textbf{Subspaces of a Vector Space:}
    \begin{itemize}
      \item A subset $W \subseteq V$ is a subspace if:
        \begin{enumerate}
          \item $0 \in W$.
          \item $v + w \in W$ for all $v, w \in W$.
          \item $a \cdot v \in W$ for all $v \in W$ and $a \in \mathbb{R}$.
        \end{enumerate}
      \item Example: The solution set of a homogeneous system $Ax = 0$ is a subspace of $\mathbb{R}^n$.
    \end{itemize}
\end{itemize}

\section*{Simplified Explanation}

\textbf{Example 1: $\mathbb{R}^2$ as a Vector Space}
For $v = (v_1, v_2)$ and $w = (w_1, w_2)$:
\[
  v + w = (v_1 + w_1, v_2 + w_2), \quad a \cdot v = (a \cdot v_1, a \cdot v_2).
\]
These operations satisfy all 8 axioms.

\textbf{Example 2: Matrices as a Vector Space}
For $A, B \in \mathbb{R}^{m \times n}$:
\[
  A + B = (A_{ij} + B_{ij}), \quad c \cdot A = (c \cdot A_{ij}),
\]
where $A_{ij}$ represents the $(i,j)$ entry of $A$. These operations satisfy the vector space axioms.

\textbf{Example 3: Homogeneous System}
The solution set of $Ax = 0$:
\begin{itemize}
  \item If $v, w \in V$, then $v + w \in V$ (closure under addition).
  \item If $v \in V$, then $a \cdot v \in V$ for $a \in \mathbb{R}$ (closure under scalar multiplication).
\end{itemize}

\section*{Conclusion}

In this lecture, we:
\begin{itemize}
  \item Defined vector spaces as a generalization of $\mathbb{R}^n$.
  \item Presented examples (e.g., $\mathbb{R}^n$, matrices, and homogeneous systems) and non-examples.
  \item Introduced subspaces and their properties.
\end{itemize}

Vector spaces provide a unified framework for understanding and manipulating linear algebra concepts. This abstraction is crucial for advanced topics such as vector calculus and linear transformations.

\end{document}
