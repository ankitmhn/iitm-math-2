\documentclass{article}
\usepackage{amsmath}
\usepackage{amssymb}
\usepackage{geometry}

\geometry{margin=1in}

\title{Lecture Summary: Equivalence and Similarity of Matrices}
\author{}
\date{}

\begin{document}

\maketitle

\section*{Source: Lec44.pdf}

\section*{Key Points}

\begin{itemize}
  \item \textbf{Equivalence of Matrices:}
    \begin{itemize}
      \item Two matrices $A$ and $B$ of size $m \times n$ are \textbf{equivalent} if there exist invertible matrices $Q$ (of size $m \times m$) and $P$ (of size $n \times n$) such that:
        \[
          B = Q \cdot A \cdot P.
        \]
      \item Equivalence can also be characterized through row and column operations:
        \begin{itemize}
          \item Row operations correspond to multiplication by $Q$.
          \item Column operations correspond to multiplication by $P$.
        \end{itemize}
      \item Equivalent matrices have the same rank.
    \end{itemize}

  \item \textbf{Properties of Equivalence:}
    \begin{itemize}
      \item \textbf{Reflexive:} $A$ is equivalent to itself since $Q$ and $P$ can be chosen as identity matrices.
      \item \textbf{Symmetric:} If $A$ is equivalent to $B$, then $B$ is equivalent to $A$:
        \[
          B = Q \cdot A \cdot P \implies A = Q^{-1} \cdot B \cdot P^{-1}.
        \]
      \item \textbf{Transitive:} If $A$ is equivalent to $B$ and $B$ is equivalent to $C$, then $A$ is equivalent to $C$:
        \[
          B = Q \cdot A \cdot P, \, C = Q' \cdot B \cdot P' \implies C = (Q' \cdot Q) \cdot A \cdot (P \cdot P').
        \]
    \end{itemize}

  \item \textbf{Similarity of Matrices:}
    \begin{itemize}
      \item Two square matrices $A$ and $B$ of size $n \times n$ are \textbf{similar} if there exists an invertible matrix $P$ of size $n \times n$ such that:
        \[
          B = P^{-1} \cdot A \cdot P.
        \]
      \item Similar matrices represent the same linear transformation under different bases.
    \end{itemize}

  \item \textbf{Properties of Similarity:}
    \begin{itemize}
      \item \textbf{Reflexive:} $A$ is similar to itself since $P$ can be chosen as the identity matrix.
      \item \textbf{Symmetric:} If $A$ is similar to $B$, then $B$ is similar to $A$:
        \[
          B = P^{-1} \cdot A \cdot P \implies A = (P^{-1})^{-1} \cdot B \cdot P^{-1}.
        \]
      \item \textbf{Transitive:} If $A$ is similar to $B$ and $B$ is similar to $C$, then $A$ is similar to $C$:
        \[
          B = P^{-1} \cdot A \cdot P, \, C = Q^{-1} \cdot B \cdot Q \implies C = (Q^{-1} \cdot P^{-1}) \cdot A \cdot (P \cdot Q).
        \]
    \end{itemize}

  \item \textbf{Why Similarity Matters:}
    \begin{itemize}
      \item Similar matrices have the same determinant, rank, characteristic polynomial, minimal polynomial, and eigenvalues.
      \item Some linear transformations can be diagonalized when expressed under an appropriate basis, simplifying their representation.
    \end{itemize}
\end{itemize}

\section*{Simplified Explanation}

\textbf{Example 1: Equivalence of Rectangular Matrices}
Let:
\[
  A =
  \begin{bmatrix}
    1 & 2 & 3 \\
    4 & 5 & 6
  \end{bmatrix}, \quad Q =
  \begin{bmatrix}
    1 & 0 \\
    0 & 2
  \end{bmatrix}, \quad P =
  \begin{bmatrix}
    1 & 0 & 0 \\
    0 & 1 & 0 \\
    0 & 0 & 3
  \end{bmatrix}.
\]
\begin{itemize}
  \item Compute $B = Q \cdot A \cdot P$:
    \[
      B =
      \begin{bmatrix}
        1 & 0 \\
        0 & 2
      \end{bmatrix}
      \begin{bmatrix}
        1 & 2 & 3 \\
        4 & 5 & 6
      \end{bmatrix}
      \begin{bmatrix}
        1 & 0 & 0 \\
        0 & 1 & 0 \\
        0 & 0 & 3
      \end{bmatrix}.
    \]
  \item The resulting $B$ is equivalent to $A$.
\end{itemize}

\textbf{Example 2: Similarity of Square Matrices}
Let:
\[
  A =
  \begin{bmatrix}
    2 & 0 \\
    0 & 3
  \end{bmatrix}, \quad P =
  \begin{bmatrix}
    1 & 1 \\
    0 & 1
  \end{bmatrix}.
\]
\begin{itemize}
  \item Compute $B = P^{-1} \cdot A \cdot P$:
    \[
      P^{-1} =
      \begin{bmatrix}
        1 & -1 \\
        0 & 1
      \end{bmatrix}, \quad B =
      \begin{bmatrix}
        1 & -1 \\
        0 & 1
      \end{bmatrix}
      \begin{bmatrix}
        2 & 0 \\
        0 & 3
      \end{bmatrix}
      \begin{bmatrix}
        1 & 1 \\
        0 & 1
      \end{bmatrix}.
    \]
  \item $A$ and $B$ are similar.
\end{itemize}

\section*{Conclusion}

In this lecture, we:
\begin{itemize}
  \item Defined equivalence and similarity of matrices.
  \item Highlighted properties and practical significance of these relations.
  \item Discussed examples demonstrating their application to linear transformations and basis changes.
\end{itemize}

Understanding equivalence and similarity helps analyze matrix representations of linear transformations under different bases, facilitating insights into their properties.

\end{document}
