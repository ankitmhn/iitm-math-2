\documentclass{article}
\usepackage{amsmath}
\usepackage{amssymb}
\usepackage{geometry}

\geometry{margin=1in}

\title{Lecture Summary: Multivariable Functions and Visualization}
\author{}
\date{}

\begin{document}

\maketitle

\section*{Source: Multivariable Functions - Visualization.pdf}

\section*{Key Points}

\begin{itemize}
  \item \textbf{Single-Variable Functions Recap:}
    \begin{itemize}
      \item Functions of the form $f(x)$ map a domain $D \subset \mathbb{R}$ to $\mathbb{R}$.
      \item Examples include:
        \begin{itemize}
          \item Linear: $f(x) = ax + b$.
          \item Polynomial: $f(x) = x^2 + x + 1$.
          \item Rational: $f(x) = \frac{x}{x^2 + 1}$.
          \item Trigonometric: $\sin(x), \cos(x)$, etc.
          \item Exponential: $e^x$.
          \item Logarithmic: $\log(x)$ (domain restricted to $x > 0$).
          \item Compositions: $\log(x^2 + 1)$, $e^{\sin(x)}$.
        \end{itemize}
      \item Key operations include addition, multiplication, and composition of functions.
    \end{itemize}

  \item \textbf{Scalar-Valued Multivariable Functions:}
    \begin{itemize}
      \item A scalar-valued multivariable function maps $f: D \subset \mathbb{R}^n \to \mathbb{R}$.
      \item Examples:
        \begin{itemize}
          \item Linear: $f(x_1, x_2, \dots, x_n) = a_1 x_1 + a_2 x_2 + \dots + a_n x_n$.
          \item Polynomial: $f(x_1, x_2, x_3) = x_1^2 + x_2^3 x_3 - x_3^6$.
          \item Rational: $f(x, y) = \frac{x}{x^2 + y^2}$ (domain excludes $x = y = 0$).
          \item Combinations: $f(x, y) = \sin(x^2 + y^2)$.
        \end{itemize}
    \end{itemize}

  \item \textbf{Vector-Valued Multivariable Functions:}
    \begin{itemize}
      \item A vector-valued multivariable function maps $f: D \subset \mathbb{R}^n \to \mathbb{R}^m$.
      \item Examples:
        \begin{itemize}
          \item $f(x, y, z) = (x^2 + y^2, y^2 + z^2, z^2 + x^2)$.
          \item $f(x, y, z) = (\sin(x), \cos(y), e^z)$.
        \end{itemize}
      \item Such functions represent vectors whose components are scalar-valued multivariable functions.
    \end{itemize}

  \item \textbf{Operations on Multivariable Functions:}
    \begin{itemize}
      \item Arithmetic: Addition, subtraction, and scalar multiplication extend naturally.
      \item Products: $f \cdot g$ is defined for scalar-valued functions.
      \item Division: $f / g$ is defined where $g \neq 0$.
      \item Composition: If $f: D \subset \mathbb{R}^n \to \mathbb{R}^m$ and $g: E \subset \mathbb{R}^m \to \mathbb{R}^p$ such that $\text{range}(f) \subseteq E$, then $g \circ f$ is well-defined.
    \end{itemize}

  \item \textbf{Visualization of Multivariable Functions:}
    \begin{itemize}
      \item Graphs of $f: \mathbb{R}^2 \to \mathbb{R}$ are surfaces in $\mathbb{R}^3$.
      \item Examples:
        \begin{itemize}
          \item Linear: $f(x, y) = ax + by$ forms a plane.
          \item Rational: $f(x, y) = \frac{xy}{x^2 + y^2}$ forms a surface with undefined behavior at $(0, 0)$.
          \item Trigonometric: $f(x, y) = \sin(x^2 + y^2)$ oscillates in concentric circles.
        \end{itemize}
    \end{itemize}

  \item \textbf{Curves in Multivariable Functions:}
    \begin{itemize}
      \item Curves are special cases of multivariable functions: $f: \mathbb{R} \to \mathbb{R}^m$.
      \item Examples:
        \begin{itemize}
          \item Helix: $\gamma(t) = (\cos(t), \sin(t), t)$ in $\mathbb{R}^3$.
          \item Circle: $\gamma(t) = (a \cos(t), b \sin(t))$ in $\mathbb{R}^2$.
        \end{itemize}
      \item Curves can be described parametrically or as sets of equations.
    \end{itemize}
\end{itemize}

\section*{Simplified Explanation}

\textbf{Scalar-Valued Functions:}
Map from $\mathbb{R}^n$ to $\mathbb{R}$, such as $f(x, y) = x^2 + y^2$.

\textbf{Vector-Valued Functions:}
Map from $\mathbb{R}^n$ to $\mathbb{R}^m$, such as $f(x, y, z) = (\sin(x), \cos(y), z^2)$.

\textbf{Visualization:}
Graphs of scalar-valued functions in $\mathbb{R}^2 \to \mathbb{R}$ form surfaces, while higher dimensions are harder to visualize.

\textbf{Curves:}
Special functions where $n = 1$, such as the helix $\gamma(t) = (\cos(t), \sin(t), t)$.

\section*{Conclusion}

In this lecture, we:
\begin{itemize}
  \item Introduced scalar- and vector-valued multivariable functions.
  \item Explored operations, including addition, multiplication, and composition.
  \item Visualized multivariable functions and curves with examples.
\end{itemize}

Multivariable functions generalize single-variable functions, forming the basis for multivariable calculus and its applications in geometry and data science.

\end{document}
