\documentclass{article}
\usepackage{amsmath}
\usepackage{amssymb}
\usepackage{geometry}

\geometry{margin=1in}

\title{Lecture Summary: Limits for Scalar-Valued Multivariable Functions}
\author{}
\date{}

\begin{document}

\maketitle

\section*{Source: Limits for Scalar-Valued Multivariable Functions.pdf}

\section*{Key Points}

\begin{itemize}
  \item \textbf{Limit of a Sequence in $\mathbb{R}^n$:}
    \begin{itemize}
      \item A sequence $\{\vec{a}_n\}$ in $\mathbb{R}^n$ converges to $\vec{a} = (a_1, a_2, \dots, a_n)$ if and only if each coordinate sequence converges:
        \[
          \lim_{n \to \infty} a_{n,i} = a_i, \quad \text{for all } i = 1, 2, \dots, n.
        \]
      \item Example:
        \[
          \left\{\left(\frac{1}{n}, n\sin\left(\frac{1}{n}\right)\right)\right\}
        \]
        converges to $(0, 1)$ because $\frac{1}{n} \to 0$ and $n\sin\left(\frac{1}{n}\right) \to 1$.
    \end{itemize}

  \item \textbf{Definition of Limit for a Multivariable Function:}
    \begin{itemize}
      \item Let $f: D \subset \mathbb{R}^n \to \mathbb{R}$ be a scalar-valued multivariable function, and $\vec{a} \in \mathbb{R}^n$. Then:
        \[
          \lim_{\vec{x} \to \vec{a}} f(\vec{x}) = L
        \]
        if for every sequence $\{\vec{a}_n\} \subset D$ with $\vec{a}_n \to \vec{a}$, we have $f(\vec{a}_n) \to L$.
      \item This definition generalizes the notion of a limit from single-variable calculus.
    \end{itemize}

  \item \textbf{Examples of Limits:}
    \begin{itemize}
      \item Example 1: For $f(x, y) = x^2 + y^2$, as $(x, y) \to (0, 0)$:
        \[
          \lim_{(x, y) \to (0, 0)} f(x, y) = 0.
        \]
      \item Example 2: For $f(x, y) = \frac{x^2 - y^2}{x^2 + y^2}$, as $(x, y) \to (0, 0)$:
        \[
          \lim_{(x, y) \to (0, 0)} f(x, y) \text{ does not exist},
        \]
        because the limit depends on the path taken (e.g., along $x = y$ versus $x = -y$).
    \end{itemize}

  \item \textbf{Properties of Limits:}
    \begin{itemize}
      \item \textbf{Linearity:} For $c \in \mathbb{R}$:
        \[
          \lim_{\vec{x} \to \vec{a}} [cf(\vec{x}) + g(\vec{x})] = c \lim_{\vec{x} \to \vec{a}} f(\vec{x}) + \lim_{\vec{x} \to \vec{a}} g(\vec{x}).
        \]
      \item \textbf{Product Rule:}
        \[
          \lim_{\vec{x} \to \vec{a}} [f(\vec{x}) \cdot g(\vec{x})] = \lim_{\vec{x} \to \vec{a}} f(\vec{x}) \cdot \lim_{\vec{x} \to \vec{a}} g(\vec{x}).
        \]
      \item \textbf{Quotient Rule:} If $\lim_{\vec{x} \to \vec{a}} g(\vec{x}) \neq 0$:
        \[
          \lim_{\vec{x} \to \vec{a}} \frac{f(\vec{x})}{g(\vec{x})} = \frac{\lim_{\vec{x} \to \vec{a}} f(\vec{x})}{\lim_{\vec{x} \to \vec{a}} g(\vec{x})}.
        \]
    \end{itemize}

  \item \textbf{Special Techniques:}
    \begin{itemize}
      \item \textbf{Substitution:} For well-behaved functions, limits can often be computed by direct substitution.
      \item \textbf{Sandwich (Squeeze) Theorem:} If $f(\vec{x}) \leq h(\vec{x}) \leq g(\vec{x})$ for all $\vec{x}$ near $\vec{a}$, and $\lim_{\vec{x} \to \vec{a}} f(\vec{x}) = \lim_{\vec{x} \to \vec{a}} g(\vec{x}) = L$, then $\lim_{\vec{x} \to \vec{a}} h(\vec{x}) = L$.
      \item \textbf{Path Dependence:} To check if a limit exists, evaluate along different paths. If the results differ, the limit does not exist.
    \end{itemize}
\end{itemize}

\section*{Simplified Explanation}

\textbf{What Are Limits for Multivariable Functions?}
Limits describe the behavior of a function $f(\vec{x})$ as $\vec{x}$ approaches a point $\vec{a}$ in $\mathbb{R}^n$.

\textbf{How to Compute Limits?}
Check if the function approaches the same value along all paths leading to $\vec{a}$. If not, the limit does not exist.

\textbf{Example:}
For $f(x, y) = \frac{x^2 - y^2}{x^2 + y^2}$, paths $x = y$ and $x = -y$ yield different limits at $(0, 0)$, so the limit does not exist.

\section*{Conclusion}

In this lecture, we:
\begin{itemize}
  \item Defined limits for scalar-valued multivariable functions.
  \item Explored techniques for computing and analyzing limits.
  \item Highlighted examples and the importance of path dependence.
\end{itemize}

Understanding limits in multivariable calculus is crucial for studying continuity, derivatives, and integrals in higher dimensions.

\end{document}
