\documentclass{article}
\usepackage{amsmath}
\usepackage{amssymb}
\usepackage{geometry}

\geometry{margin=1in}

\title{Lecture Summary: Tangent Hyperplanes and Linear Approximations}
\author{}
\date{}

\begin{document}

\maketitle

\section*{Source: Week 11 - Lec 03.pdf}

\section*{Key Points}

\begin{itemize}
  \item \textbf{Tangent Hyperplanes:}
    \begin{itemize}
      \item A tangent hyperplane generalizes the concept of a tangent plane (in $\mathbb{R}^3$) to higher dimensions.
      \item For a scalar-valued function $f: D \subset \mathbb{R}^n \to \mathbb{R}$, the tangent hyperplane at $\vec{a} \in \mathbb{R}^n$ is the affine flat that contains all tangent lines at $\vec{a}$.

      \item \textbf{Equation of the Tangent Hyperplane:}
        \[
          z = f(\vec{a}) + \sum_{i=1}^n \frac{\partial f}{\partial x_i}(\vec{a})(x_i - a_i),
        \]
        or equivalently,
        \[
          z = f(\vec{a}) + \nabla f(\vec{a}) \cdot (\vec{x} - \vec{a}),
        \]
        where $\nabla f$ is the gradient of $f$.
    \end{itemize}

  \item \textbf{Interpretation in $\mathbb{R}^3$:}
    \begin{itemize}
      \item For $f(x, y)$, the tangent plane at $(a, b)$ is given by:
        \[
          z = f(a, b) + \frac{\partial f}{\partial x}(a, b)(x - a) + \frac{\partial f}{\partial y}(a, b)(y - b).
        \]
      \item Geometrically, all tangent lines at $(a, b)$ lie on this plane.
    \end{itemize}

  \item \textbf{Linear Approximation:}
    \begin{itemize}
      \item The tangent hyperplane provides the best linear approximation to $f$ near $\vec{a}$.
      \item Define $L_f(\vec{x})$ as:
        \[
          L_f(\vec{x}) = f(\vec{a}) + \nabla f(\vec{a}) \cdot (\vec{x} - \vec{a}).
        \]
      \item This is the linear approximation of $f$ around $\vec{a}$.
    \end{itemize}

  \item \textbf{Examples:}
    \begin{itemize}
      \item \textbf{Example 1:} $f(x, y) = x + y$ at $(1, 1)$:
        \[
          \nabla f = (1, 1), \quad z = 2 + 1(x - 1) + 1(y - 1).
        \]
        Simplify to:
        \[
          z = x + y.
        \]

      \item \textbf{Example 2:} $f(x, y) = xy$ at $(1, 1)$:
        \[
          \nabla f = (y, x), \quad z = 1 + 1(x - 1) + 1(y - 1).
        \]
        Simplify to:
        \[
          z = x + y - 1.
        \]

      \item \textbf{Example 3:} $f(x, y, z) = x^2 + y^2 + z^2$ at $(2, 3, -1)$:
        \[
          \nabla f = (2x, 2y, 2z), \quad \nabla f(2, 3, -1) = (4, 6, -2).
        \]
        Tangent hyperplane:
        \[
          z = 14 + 4(x - 2) + 6(y - 3) - 2(z + 1).
        \]
        Simplify to:
        \[
          z = 14 + 4x + 6y - 2z - 32 \implies z = 4x + 6y - 2z - 18.
        \]
    \end{itemize}

  \item \textbf{Conditions for Existence:}
    \begin{itemize}
      \item The gradient $\nabla f$ must exist and be continuous in a neighborhood of $\vec{a}$.
      \item If $\nabla f$ is discontinuous or undefined, the tangent hyperplane may not exist.
    \end{itemize}
\end{itemize}

\section*{Simplified Explanation}

\textbf{Tangent Hyperplanes:}
The tangent hyperplane is a generalization of the tangent plane for higher dimensions. It contains all tangent lines at a point and approximates the function locally.

\textbf{Equation:}
For $f(x, y)$ at $(a, b)$:
\[
  z = f(a, b) + \frac{\partial f}{\partial x}(a, b)(x - a) + \frac{\partial f}{\partial y}(a, b)(y - b).
\]

\textbf{Example:}
For $f(x, y) = x + y$ at $(1, 1)$:
\[
  z = x + y.
\]

\section*{Conclusion}

In this lecture, we:
\begin{itemize}
  \item Defined tangent hyperplanes and their role in multivariable calculus.
  \item Derived their equations and discussed their geometric significance.
  \item Demonstrated their use in linear approximations.
\end{itemize}

Tangent hyperplanes are essential for approximating and understanding the behavior of multivariable functions near a point.

\end{document}
