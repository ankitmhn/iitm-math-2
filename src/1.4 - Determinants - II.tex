\documentclass{article}
\usepackage{amsmath}
\usepackage{amssymb}
\usepackage{geometry}

\geometry{margin=1in}

\title{Lecture Summary: Determinants (Part 2)}
\author{}
\date{}

\begin{document}

\maketitle

\section*{Source: Lec 20.pdf}

\section*{Key Points}

\begin{itemize}
  \item \textbf{Review of Determinants (Part 1):}
    \begin{itemize}
      \item Determinants for $1 \times 1$, $2 \times 2$, and $3 \times 3$ matrices were defined.
      \item For a $3 \times 3$ matrix, the determinant is calculated using expansion along the first row:
        \[
          \det(A) = a_{11}
          \begin{vmatrix}
            a_{22} & a_{23} \\
            a_{32} & a_{33}
          \end{vmatrix}
          - a_{12}
          \begin{vmatrix}
            a_{21} & a_{23} \\
            a_{31} & a_{33}
          \end{vmatrix}
          + a_{13}
          \begin{vmatrix}
            a_{21} & a_{22} \\
            a_{31} & a_{32}
          \end{vmatrix}.
        \]
    \end{itemize}

  \item \textbf{Special Matrices and Properties:}
    \begin{itemize}
      \item \textbf{Upper Triangular and Lower Triangular Matrices:}
        - Determinant is the product of diagonal elements.
        - Example:
        \[
          A =
          \begin{bmatrix}
            2 & 4 & 3 \\
            0 & 8 & 7 \\
            0 & 0 & 9
          \end{bmatrix},
          \quad
          \det(A) = 2 \cdot 8 \cdot 9 = 144.
        \]

      \item \textbf{Transpose of a Matrix:}
        - Transpose reflects the matrix about its diagonal.
        - Determinant remains unchanged:
        \[
          \det(A^T) = \det(A).
        \]
    \end{itemize}

  \item \textbf{Minors and Cofactors:}
    \begin{itemize}
      \item \textbf{Minor:} Determinant of the submatrix obtained by deleting the $i$-th row and $j$-th column. Denoted by $M_{ij}$.
      \item \textbf{Cofactor:} Minor adjusted by a sign factor:
        \[
          C_{ij} = (-1)^{i+j} M_{ij}.
        \]
      \item Example:
        For $A =
        \begin{bmatrix}
          a_{11} & a_{12} & a_{13} \\
          a_{21} & a_{22} & a_{23} \\
          a_{31} & a_{32} & a_{33}
        \end{bmatrix}$:
        \[
          C_{11} = M_{11}, \quad C_{12} = -M_{12}.
        \]
    \end{itemize}

  \item \textbf{Inductive Definition of Determinants:}
    \begin{itemize}
      \item Determinants for $n \times n$ matrices are defined using determinants of $(n-1) \times (n-1)$ matrices.
      \item Formula for expansion along the first row:
        \[
          \det(A) = \sum_{j=1}^n (-1)^{1+j} a_{1j} M_{1j} = \sum_{j=1}^n a_{1j} C_{1j}.
        \]
      \item This process generalizes to any matrix size.
    \end{itemize}
\end{itemize}

\section*{Simplified Explanation}

\textbf{Example: Determinant of a $4 \times 4$ Matrix}
For $A =
\begin{bmatrix}
  a_{11} & a_{12} & a_{13} & a_{14} \\
  a_{21} & a_{22} & a_{23} & a_{24} \\
  a_{31} & a_{32} & a_{33} & a_{34} \\
  a_{41} & a_{42} & a_{43} & a_{44}
\end{bmatrix}$:
\[
  \det(A) = a_{11}
  \begin{vmatrix}
    a_{22} & a_{23} & a_{24} \\
    a_{32} & a_{33} & a_{34} \\
    a_{42} & a_{43} & a_{44}
  \end{vmatrix}
  - a_{12}
  \begin{vmatrix}
    a_{21} & a_{23} & a_{24} \\
    a_{31} & a_{33} & a_{34} \\
    a_{41} & a_{43} & a_{44}
  \end{vmatrix}
  + \cdots.
\]

\textbf{Identity Matrix Determinant:}
For the identity matrix $I_n$ (with diagonal entries 1 and off-diagonal entries 0):
\[
  \det(I_n) = 1.
\]

\section*{Conclusion}

In this lecture, we:
\begin{itemize}
  \item Reviewed determinants for small matrices and explored properties of special matrices.
  \item Defined minors and cofactors, essential for expanding determinants to larger matrices.
  \item Generalized determinants using an inductive approach, allowing computation for any $n \times n$ matrix.
\end{itemize}

Determinants play a crucial role in solving systems of linear equations, finding matrix inverses, and applications in higher mathematics like calculus.

\end{document}
