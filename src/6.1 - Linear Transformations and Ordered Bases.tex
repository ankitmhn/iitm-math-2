\documentclass{article}
\usepackage{amsmath}
\usepackage{amssymb}
\usepackage{geometry}

\geometry{margin=1in}

\title{Lecture Summary: Linear Transformations and Ordered Bases}
\author{}
\date{}

\begin{document}

\maketitle

\section*{Source: Lec41.pdf}

\section*{Key Points}

\begin{itemize}
  \item \textbf{Linear Transformations: Recap and Basis Dependency:}
    \begin{itemize}
      \item A linear transformation $f: V \to W$ satisfies:
        \[
          f(u + v) = f(u) + f(v), \quad f(c \cdot u) = c \cdot f(u),
        \]
        for $u, v \in V$ and scalar $c \in \mathbb{R}$.
      \item The action of $f$ is fully determined by its values on the basis vectors of $V$.
    \end{itemize}

  \item \textbf{Isomorphism Using Basis:}
    \begin{itemize}
      \item If $V$ is an $n$-dimensional vector space with basis $\{v_1, v_2, \dots, v_n\}$:
        \begin{enumerate}
          \item Define $f(v_i) = e_i$ (the standard basis for $\mathbb{R}^n$).
          \item Extend this mapping linearly to $V$ by:
            \[
              f\left(\sum_{i=1}^n c_i v_i\right) = \sum_{i=1}^n c_i e_i.
            \]
        \end{enumerate}
      \item $f$ is an isomorphism, meaning it is both one-to-one and onto.
    \end{itemize}

  \item \textbf{Matrix Representation of Linear Transformations:}
    \begin{itemize}
      \item For $f: V \to W$, let $\beta = \{v_1, v_2, \dots, v_n\}$ be an ordered basis for $V$ and $\gamma = \{w_1, w_2, \dots, w_m\}$ for $W$.
      \item Represent $f(v_j)$ as a linear combination of $\{w_1, w_2, \dots, w_m\}$:
        \[
          f(v_j) = \sum_{i=1}^m a_{ij} w_i.
        \]
      \item The coefficients $a_{ij}$ form the $i$th row and $j$th column of the matrix representation of $f$.
    \end{itemize}

  \item \textbf{Example: Linear Transformation on $\mathbb{R}^2$:}
    \begin{itemize}
      \item Define $f(x, y) = (2x, y)$.
      \item With standard basis $\{(1, 0), (0, 1)\}$, compute:
        \[
          f(1, 0) = (2, 0), \quad f(0, 1) = (0, 1).
        \]
      \item Matrix representation:
        \[
          A =
          \begin{bmatrix}
            2 & 0 \\
            0 & 1
          \end{bmatrix}.
        \]
    \end{itemize}

  \item \textbf{Changing Ordered Basis:}
    \begin{itemize}
      \item Changing the basis changes the matrix representation.
      \item Example: If $\beta = \{(1, 0), (1, 1)\}$, then:
        \[
          f(1, 0) = (2, 0), \quad f(1, 1) = (2, 1).
        \]
      \item Matrix representation becomes:
        \[
          A =
          \begin{bmatrix}
            2 & 1 \\
            0 & 1
          \end{bmatrix}.
        \]
      \item The same linear transformation yields different matrices for different bases.
    \end{itemize}

  \item \textbf{Bijection Between Linear Transformations and Matrices:}
    \begin{itemize}
      \item For fixed ordered bases $\beta$ and $\gamma$, there is a bijection between linear transformations $f: V \to W$ and $m \times n$ matrices.
      \item The matrix $A$ encodes the coefficients of $f(v_j)$ expressed in terms of $\gamma$.
    \end{itemize}
\end{itemize}

\section*{Simplified Explanation}

\textbf{Example 1: Linear Transformation on $\mathbb{R}^3$}
Let $W = \{(x, y, z) \in \mathbb{R}^3 \mid x + y + z = 0\}$ and $V = \mathbb{R}^2$:
\begin{itemize}
  \item Basis for $W$: $\{(-1, 1, 0), (-1, 0, 1)\}$.
  \item Define $f(-1, 1, 0) = (1, 0)$, $f(-1, 0, 1) = (0, 1)$.
  \item Matrix representation (standard basis for $V$):
    \[
      A =
      \begin{bmatrix}
        1 & 0 \\
        0 & 1
      \end{bmatrix}.
    \]
\end{itemize}

\section*{Conclusion}

In this lecture, we:
\begin{itemize}
  \item Explored linear transformations and their dependency on ordered bases.
  \item Developed matrix representations and analyzed their changes with basis choice.
  \item Highlighted the bijection between linear transformations and matrices for fixed bases.
\end{itemize}

This framework unifies the concepts of linear algebra, emphasizing the interplay between transformations, bases, and matrix representations.

\end{document}
