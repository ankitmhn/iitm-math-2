\documentclass{article}
\usepackage{amsmath}
\usepackage{amssymb}
\usepackage{geometry}

\geometry{margin=1in}

\title{Lecture Summary: Null Space and Basis – Part 2}
\author{}
\date{}

\begin{document}

\maketitle

\section*{Source: Lec37.pdf}

\section*{Key Points}

\begin{itemize}
  \item \textbf{Recap: Null Space and Nullity:}
    \begin{itemize}
      \item The null space of an $m \times n$ matrix $A$ is:
        \[
          \text{Null}(A) = \{x \in \mathbb{R}^n \mid Ax = 0\}.
        \]
      \item Nullity is the dimension of the null space.
      \item Rank-Nullity Theorem:
        \[
          \text{rank}(A) + \text{nullity}(A) = n.
        \]
    \end{itemize}

  \item \textbf{Procedure to Find the Null Space and Basis:}
    \begin{enumerate}
      \item Form the augmented matrix $[A \mid 0]$.
      \item Perform row reduction to bring $A$ to reduced row echelon form (RREF).
      \item Identify dependent and independent variables:
        \begin{itemize}
          \item Columns with leading 1s correspond to dependent variables.
          \item Other columns correspond to independent variables.
        \end{itemize}
      \item Assign parameters to independent variables and solve for dependent variables in terms of the parameters.
      \item Basis for the null space consists of vectors generated by setting one parameter to 1 and the others to 0.
    \end{enumerate}

  \item \textbf{Example: $3 \times 4$ Matrix}
    \begin{itemize}
      \item Given:
        \[
          A =
          \begin{bmatrix}
            1 & 2 & 0 & 3 \\
            2 & 3 & 0 & 3 \\
            1 & 1 & 1 & 2
          \end{bmatrix}.
        \]
      \item Row reduction to RREF:
        \[
          \begin{bmatrix}
            1 & 0 & 0 & -3 \\
            0 & 1 & 0 & 3 \\
            0 & 0 & 1 & 2
          \end{bmatrix}.
        \]
      \item Dependent variables: $x_1, x_2, x_3$.
      \item Independent variable: $x_4$.
      \item Solve for $x_1, x_2, x_3$ in terms of $x_4 = t$:
        \[
          x_1 = 3t, \quad x_2 = -3t, \quad x_3 = -2t.
        \]
      \item General solution:
        \[
          \{(3t, -3t, -2t, t) \mid t \in \mathbb{R}\}.
        \]
      \item Basis vector for the null space:
        \[
          (3, -3, -2, 1).
        \]
      \item Nullity = 1 (one independent variable).
    \end{itemize}

  \item \textbf{Rank and Nullity Relationship:}
    \begin{itemize}
      \item Rank is the number of non-zero rows in the row-reduced matrix.
      \item Nullity is the number of free (independent) variables.
      \item For $A$ with 4 columns:
        \[
          \text{rank}(A) = 3, \quad \text{nullity}(A) = 1, \quad \text{rank}(A) + \text{nullity}(A) = 4.
        \]
    \end{itemize}
\end{itemize}

\section*{Simplified Explanation}

\textbf{Example 1: Null Space and Basis in $\mathbb{R}^3$}
For $A$:
\[
  \begin{bmatrix}
    1 & 1 & 1 \\
    2 & 2 & 2 \\
    3 & 3 & 3
  \end{bmatrix},
\]
RREF:
\[
  \begin{bmatrix}
    1 & 1 & 1 \\
    0 & 0 & 0 \\
    0 & 0 & 0
  \end{bmatrix}.
\]
\begin{itemize}
  \item $x_1$ is dependent, $x_2, x_3$ are independent.
  \item Solve for $x_1$:
    \[
      x_1 = -x_2 - x_3.
    \]
  \item Basis vectors:
    \[
      (-1, 1, 0), \quad (-1, 0, 1).
    \]
  \item Nullity = 2.
\end{itemize}

\textbf{Example 2: Basis Verification Using Determinants}
Given vectors $(1, 2, 3)$, $(0, 1, 2)$, and $(1, 3, 0)$ in $\mathbb{R}^3$:
\begin{itemize}
  \item Form matrix $A$ with these as columns:
    \[
      A =
      \begin{bmatrix}
        1 & 0 & 1 \\
        2 & 1 & 3 \\
        3 & 2 & 0
      \end{bmatrix}.
    \]
  \item Compute determinant:
    \[
      \det(A) = -5 \neq 0.
    \]
  \item Non-zero determinant implies linear independence, so these vectors form a basis.
\end{itemize}

\section*{Conclusion}

In this lecture, we:
\begin{itemize}
  \item Computed null space and nullity using Gaussian elimination.
  \item Verified rank and nullity using the rank-nullity theorem.
  \item Connected linear independence to determinant checks for basis validation.
\end{itemize}

Understanding null spaces and bases enhances our ability to analyze matrix transformations and solve systems of equations.

\end{document}
