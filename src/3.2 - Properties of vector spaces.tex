\documentclass{article}
\usepackage{amsmath}
\usepackage{amssymb}
\usepackage{geometry}

\geometry{margin=1in}

\title{Lecture Summary: Properties of Vector Spaces}
\author{}
\date{}

\begin{document}

\maketitle

\section*{Source: Lec28.pdf}

\section*{Key Points}

\begin{itemize}
  \item \textbf{Cancellation Law:}
    \begin{itemize}
      \item If $v_1, v_2, v_3 \in V$ and $v_1 + v_3 = v_2 + v_3$, then $v_1 = v_2$.
      \item This follows directly from the vector space axioms using the existence of additive inverses and associativity of addition.
    \end{itemize}

  \item \textbf{Uniqueness of the Zero Vector:}
    \begin{itemize}
      \item The zero vector $0 \in V$, as defined in the axioms, is unique.
      \item Proof: Suppose $w$ also satisfies $v + w = v$ for all $v \in V$. Then $w = 0$ by the cancellation law.
    \end{itemize}

  \item \textbf{Uniqueness of the Additive Inverse:}
    \begin{itemize}
      \item For each $v \in V$, there exists a unique $-v$ such that $v + (-v) = 0$.
      \item Proof: If another vector $w$ satisfies $v + w = 0$, then $w = -v$ by the cancellation law.
    \end{itemize}

  \item \textbf{Additional Properties:}
    \begin{itemize}
      \item $0 \cdot v = 0$ for all $v \in V$.
      \item $-c \cdot v = -(c \cdot v) = c \cdot (-v)$ for all $v \in V$ and scalars $c$.
      \item $c \cdot 0 = 0$ for all scalars $c$.
    \end{itemize}

  \item \textbf{Real-Life Application - Grocery Shop Example:}
    \begin{itemize}
      \item Quantities of items (e.g., rice, dal, oil, biscuits, soap) are treated as vectors.
      \item Addition corresponds to summing stocks or demands, and scalar multiplication adjusts quantities by a factor.
      \item Negative quantities represent demand, while positive quantities represent supply.
      \item This vector space behaves like $\mathbb{R}^5$.
    \end{itemize}

  \item \textbf{Affine Flat Example:}
    \begin{itemize}
      \item Consider a plane $V$ parallel to the $XY$ plane.
      \item Scalar multiplication and addition involve projections onto the $XY$ plane, performing operations in $\mathbb{R}^2$, and projecting back to $V$.
      \item Visualization: Use arrows from a fixed point (e.g., the intersection of the $Z$-axis with $V$) to perform operations geometrically.
    \end{itemize}
\end{itemize}

\section*{Simplified Explanation}

\textbf{Cancellation Law Example:}
For $v_1 + v_3 = v_2 + v_3$, subtract $v_3$ from both sides to get $v_1 = v_2$. This demonstrates the use of the axioms without additional structure.

\textbf{Grocery Shop Vector Space:}
Items like rice (kg), dal (kg), oil (liters), biscuits (packets), and soap bars form vectors:
\[
  \begin{bmatrix}
    q_{\text{rice}} \\
    q_{\text{dal}} \\
    q_{\text{oil}} \\
    q_{\text{biscuits}} \\
    q_{\text{soap}}
  \end{bmatrix}.
\]
Addition sums stocks or demands across items, while scalar multiplication adjusts quantities. Negative quantities represent demand.

\textbf{Affine Flat:}
Operations on a plane parallel to $XY$ involve projecting points to $XY$, performing $\mathbb{R}^2$ operations, and projecting back. For example:
\begin{itemize}
  \item Scalar multiplication stretches or shrinks vectors.
  \item Addition uses the parallelogram law in $\mathbb{R}^2$ before projecting back to $V$.
\end{itemize}

\section*{Conclusion}

In this lecture, we:
\begin{itemize}
  \item Proved key properties of vector spaces, including the uniqueness of the zero vector and additive inverses.
  \item Demonstrated practical examples, such as the grocery shop vector space and affine flats, to illustrate abstract concepts.
  \item Highlighted the geometric intuition behind vector space operations.
\end{itemize}

These properties deepen our understanding of the structure and utility of vector spaces, preparing us for further exploration of linear transformations and geometry.

\end{document}
