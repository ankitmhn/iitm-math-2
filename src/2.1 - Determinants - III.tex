\documentclass{article}
\usepackage{amsmath}
\usepackage{amssymb}
\usepackage{geometry}

\geometry{margin=1in}

\title{Lecture Summary: Determinants (Part 3)}
\author{Ankit Mohan \\ 24f1002261@ds.study.iitm.ac.in}

\begin{document}

\maketitle

\section*{Source: Lec21.pdf}

\section*{Key Points}

\begin{itemize}
  \item \textbf{Review of Determinants (Parts 1 and 2):}
    \begin{itemize}
      \item Determinants were defined inductively for $1 \times 1$, $2 \times 2$, and $3 \times 3$ matrices.
      \item Expansion along the first row was used to compute determinants, with minors and cofactors as key components:
        \[
          \text{Cofactor: } C_{ij} = (-1)^{i+j} M_{ij}.
        \]
    \end{itemize}

  \item \textbf{Expansion Along Any Row or Column:}
    \begin{itemize}
      \item Determinants can be computed by expanding along any row or column.
      \item Formula for expansion along the $i$th row:
        \[
          \det(A) = \sum_{j=1}^n (-1)^{i+j} a_{ij} M_{ij}.
        \]
      \item Formula for expansion along the $j$th column:
        \[
          \det(A) = \sum_{i=1}^n (-1)^{i+j} a_{ij} M_{ij}.
        \]
    \end{itemize}

  \item \textbf{Properties of Determinants:}
    \begin{itemize}
      \item \textbf{Multiplication:}
        \[
          \det(AB) = \det(A) \cdot \det(B).
        \]
      \item \textbf{Powers and Inverses:}
        \[
          \det(A^n) = (\det(A))^n, \quad \det(A^{-1}) = \frac{1}{\det(A)}.
        \]
      \item \textbf{Similarity Transformation:}
        \[
          \det(P^{-1}AP) = \det(A).
        \]
      \item \textbf{Transpose:}
        \[
          \det(A^T) = \det(A), \quad \det(A^T A) = (\det(A))^2.
        \]
      \item \textbf{Row/Column Operations:}
        \begin{itemize}
          \item Swapping two rows or columns changes the determinant's sign.
          \item Adding a multiple of one row (or column) to another does not change the determinant.
          \item Multiplying a row or column by $t$ scales the determinant by $t$.
          \item Multiplying the entire matrix by $t$ scales the determinant by $t^n$, where $n$ is the matrix size.
        \end{itemize}
    \end{itemize}

  \item \textbf{Computational Tips:}
    \begin{itemize}
      \item Determinants of matrices with a zero row or column are $0$.
      \item Determinants of matrices with a row (or column) that is a linear combination of others are $0$.
      \item Expand along a row or column with the most zeros for easier computation.
    \end{itemize}
\end{itemize}

\section*{Simplified Explanation}

\textbf{Example: Determinant by Expansion Along the Second Row}
For $A =
\begin{bmatrix}
  a_{11} & a_{12} & a_{13} \\
  a_{21} & a_{22} & a_{23} \\
  a_{31} & a_{32} & a_{33}
\end{bmatrix}$:
\[
  \det(A) = -a_{21}
  \begin{vmatrix}
    a_{12} & a_{13} \\
    a_{32} & a_{33}
  \end{vmatrix}
  + a_{22}
  \begin{vmatrix}
    a_{11} & a_{13} \\
    a_{31} & a_{33}
  \end{vmatrix}
  - a_{23}
  \begin{vmatrix}
    a_{11} & a_{12} \\
    a_{31} & a_{32}
  \end{vmatrix}.
\]

\textbf{Example: Multiplying a Matrix by $t$}
If $A$ is an $n \times n$ matrix and every element is multiplied by $t$:
\[
  \det(tA) = t^n \det(A).
\]

\section*{Conclusion}

In this lecture, we:
\begin{itemize}
  \item Expanded determinants along any row or column, showing that the method generalizes beyond the first row.
  \item Explored additional determinant properties, including behavior under row/column operations, matrix transposes, and scalar multiplication.
  \item Reviewed computational tips to simplify determinant calculations.
\end{itemize}

In the next lecture, we will use determinants to solve systems of linear equations, showcasing their practical applications.

\end{document}
