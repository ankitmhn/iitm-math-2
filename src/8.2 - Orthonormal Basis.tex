\documentclass{article}
\usepackage{amsmath}
\usepackage{amssymb}
\usepackage{geometry}

\geometry{margin=1in}

\title{Lecture Summary: Orthonormal Basis}
\author{}
\date{}

\begin{document}

\maketitle

\section*{Source: What is an orthonormal basis (1).pdf}

\section*{Key Points}

\begin{itemize}
  \item \textbf{Orthonormal Sets:}
    \begin{itemize}
      \item A set of vectors $\{v_1, v_2, \dots, v_k\}$ in an inner product space is orthonormal if:
        \begin{enumerate}
          \item The vectors are mutually orthogonal:
            \[
              \langle v_i, v_j \rangle = 0 \quad \text{for } i \neq j.
            \]
          \item Each vector has norm 1:
            \[
              \|v_i\| = 1 \implies \langle v_i, v_i \rangle = 1.
            \]
        \end{enumerate}
      \item Example in $\mathbb{R}^4$ with the standard dot product:
        \[
          \left\{\frac{1}{\sqrt{3}}(1, 1, 1, 0), \frac{1}{\sqrt{42}}(2, 1, 1, 6), \frac{1}{3}(2, 0, 2, -1)\right\}
        \]
        is an orthonormal set.
    \end{itemize}

  \item \textbf{Orthonormal Basis:}
    \begin{itemize}
      \item An orthonormal basis is an orthonormal set that is also a basis for the vector space.
      \item Equivalently, it is an orthogonal basis where each vector has norm 1.
      \item The standard basis in $\mathbb{R}^n$ with the dot product is an example of an orthonormal basis.
      \item Example in $\mathbb{R}^3$:
        \[
          \left\{\frac{1}{3}(1, 2, 2), \frac{1}{3}(-2, -1, 2), \frac{1}{3}(2, -2, 1)\right\}.
        \]
    \end{itemize}

  \item \textbf{Constructing an Orthonormal Basis:}
    \begin{itemize}
      \item From an orthogonal set $\{v_1, v_2, \dots, v_k\}$, create an orthonormal set $\{u_1, u_2, \dots, u_k\}$ by dividing each vector by its norm:
        \[
          u_i = \frac{v_i}{\|v_i\|}.
        \]
      \item This method preserves orthogonality while ensuring each vector has norm 1.
      \item Example in $\mathbb{R}^2$:
        \[
          \gamma = \left\{(1, 3), (-3, 1)\right\} \quad \text{becomes} \quad \beta = \left\{\frac{1}{\sqrt{10}}(1, 3), \frac{1}{\sqrt{10}}(-3, 1)\right\}.
        \]
    \end{itemize}

  \item \textbf{Importance of Orthonormal Bases:}
    \begin{itemize}
      \item Given an orthonormal basis $\{v_1, v_2, \dots, v_n\}$ for $V$, any vector $v \in V$ can be uniquely expressed as:
        \[
          v = c_1 v_1 + c_2 v_2 + \dots + c_n v_n,
        \]
        where:
        \[
          c_i = \langle v, v_i \rangle.
        \]
      \item This simplifies the computation of coefficients in a linear combination, avoiding the need to solve systems of equations.
    \end{itemize}
\end{itemize}

\section*{Simplified Explanation}

\textbf{Orthonormal Sets:}
A set of vectors is orthonormal if they are perpendicular (orthogonal) and have a length (norm) of 1.

\textbf{Orthonormal Basis:}
An orthonormal basis is a maximal orthonormal set that spans the entire vector space. For example, the standard basis in $\mathbb{R}^n$ is orthonormal.

\textbf{Constructing an Orthonormal Basis:}
Divide each vector in an orthogonal set by its norm to make it orthonormal.

\textbf{Applications:}
With an orthonormal basis, coefficients in a linear combination can be computed as inner products.

\section*{Conclusion}

In this lecture, we:
\begin{itemize}
  \item Defined orthonormal sets and orthonormal bases.
  \item Showed how to construct an orthonormal basis from an orthogonal set.
  \item Highlighted the computational advantages of using orthonormal bases.
\end{itemize}

Orthonormal bases streamline vector space computations, making them crucial in linear algebra and applications like data science.

\end{document}
