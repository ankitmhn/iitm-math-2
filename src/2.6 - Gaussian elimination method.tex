\documentclass{article}
\usepackage{amsmath}
\usepackage{amssymb}
\usepackage{geometry}

\geometry{margin=1in}

\title{Lecture Summary: Gaussian Elimination Method}
\author{}
\date{}

\begin{document}

\maketitle

\section*{Source: Lec26.pdf}

\section*{Key Points}

\begin{itemize}
  \item \textbf{Purpose of Gaussian Elimination:}
    \begin{itemize}
      \item Solve systems of linear equations $Ax = b$.
      \item Determine if a solution exists.
      \item Find the determinant of a square matrix.
      \item Compute the inverse of an invertible square matrix.
    \end{itemize}

  \item \textbf{Augmented Matrix:}
    \begin{itemize}
      \item The system $Ax = b$ is written as an augmented matrix $[A|b]$.
      \item $A$ is an $m \times n$ matrix of coefficients, and $b$ is an $m \times 1$ column vector of constants.
      \item The augmented matrix $[A|b]$ is of size $m \times (n+1)$.
    \end{itemize}

  \item \textbf{Gaussian Elimination Steps:}
    \begin{enumerate}
      \item Form the augmented matrix $[A|b]$.
      \item Use elementary row operations to bring the matrix to Row Echelon Form (REF):
        \begin{itemize}
          \item Create leading 1s in the rows.
          \item Eliminate entries below the leading 1s.
        \end{itemize}
      \item Further reduce to Reduced Row Echelon Form (RREF) by:
        \begin{itemize}
          \item Making all entries above and below each leading 1 equal to 0.
        \end{itemize}
    \end{enumerate}

  \item \textbf{Interpreting Solutions:}
    \begin{itemize}
      \item If the last row of $[A|b]$ is $[0 \cdots 0 | c]$ where $c \neq 0$, the system is inconsistent (no solution).
      \item If the system is consistent:
        \begin{enumerate}
          \item Identify dependent variables corresponding to leading 1s.
          \item Assign arbitrary values to independent variables (free variables).
          \item Solve for dependent variables by back substitution.
        \end{enumerate}
    \end{itemize}

  \item \textbf{Homogeneous Systems ($Ax = 0$):}
    \begin{itemize}
      \item $x = 0$ (trivial solution) is always a solution.
      \item If $A$ has more variables than equations ($n > m$), there are infinitely many solutions, parameterized by free variables.
    \end{itemize}

  \item \textbf{Advantages of Gaussian Elimination:}
    \begin{itemize}
      \item Algorithmic and efficient for solving linear systems.
      \item Avoids repeated determinant calculations (as in Cramer’s Rule).
      \item Simplifies finding matrix inverses and determinants.
    \end{itemize}
\end{itemize}

\section*{Simplified Explanation}

\textbf{Example 1: Augmented Matrix and Solution}
System of equations:
\[
  3x_1 + 2x_2 + x_3 = 6, \quad x_1 + x_2 = 2, \quad 7x_2 + x_3 + x_4 = 8.
\]
Augmented matrix:
\[
  [A|b] =
  \begin{bmatrix}
    3 & 2 & 1 & 0 & 6 \\
    1 & 1 & 0 & 0 & 2 \\
    0 & 7 & 1 & 1 & 8
  \end{bmatrix}.
\]
Steps:
\begin{enumerate}
  \item Divide $R_1$ by 3 to make the first pivot 1.
  \item Eliminate entries below the pivot in the first column.
  \item Repeat for subsequent columns to reach REF:
    \[
      \begin{bmatrix}
        1 & 0 & 0 & 0 & 1 \\
        0 & 1 & 0 & 0 & 1 \\
        0 & 0 & 1 & 1 & 1
      \end{bmatrix}.
    \]
\end{enumerate}
Interpretation:
\[
  x_1 = 1, \quad x_2 = 1, \quad x_3 + x_4 = 1.
\]
General solution:
\[
  x_1 = 1, \quad x_2 = 1, \quad x_3 = 1 - c, \quad x_4 = c \quad (c \in \mathbb{R}).
\]

\textbf{Example 2: Homogeneous System}
For $Ax = 0$, with $A$ having more variables than equations:
\[
  \text{Infinitely many solutions exist, parameterized by free variables.}
\]

\section*{Conclusion}

In this lecture, we:
\begin{itemize}
  \item Introduced Gaussian elimination as a systematic method for solving $Ax = b$.
  \item Defined the augmented matrix and illustrated its use in solving linear systems.
  \item Demonstrated Gaussian elimination for both consistent and inconsistent systems.
  \item Explored the special case of homogeneous systems and their solution properties.
\end{itemize}

Gaussian elimination is a versatile and efficient tool for solving linear systems, computing determinants, and finding matrix inverses.

\end{document}
