\documentclass{article}
\usepackage{amsmath}
\usepackage{amssymb}
\usepackage{geometry}

\geometry{margin=1in}

\title{Lecture Summary: Linear Dependence}
\author{}
\date{}

\begin{document}

\maketitle

\section*{Source: Lec29.pdf}

\section*{Key Points}

\begin{itemize}
  \item \textbf{Definition of Linear Dependence:}
    \begin{itemize}
      \item A set of vectors $v_1, v_2, \dots, v_n$ in a vector space $V$ is \textbf{linearly dependent} if there exist scalars $a_1, a_2, \dots, a_n$, not all zero, such that:
        \[
          a_1 v_1 + a_2 v_2 + \dots + a_n v_n = 0.
        \]
      \item Equivalently, the zero vector is a linear combination of $v_1, v_2, \dots, v_n$ with at least one non-zero coefficient.
    \end{itemize}

  \item \textbf{Key Observations:}
    \begin{itemize}
      \item If two vectors are linearly dependent, one is a scalar multiple of the other, and they lie on the same line.
      \item If three vectors in $\mathbb{R}^3$ are linearly dependent, they lie on the same plane.
      \item If a set of vectors is linearly dependent, any superset of that set is also linearly dependent.
    \end{itemize}

  \item \textbf{Linear Combinations:}
    \begin{itemize}
      \item A vector $v$ is a linear combination of $v_1, v_2, \dots, v_n$ if:
        \[
          v = a_1 v_1 + a_2 v_2 + \dots + a_n v_n,
        \]
        where $a_1, a_2, \dots, a_n$ are scalars.
      \item Example:
        \[
          2(1, 2) + (2, 1) = (4, 5).
        \]
    \end{itemize}

  \item \textbf{Testing for Linear Dependence:}
    \begin{itemize}
      \item Form the equation:
        \[
          a_1 v_1 + a_2 v_2 + \dots + a_n v_n = 0.
        \]
      \item Solve for $a_1, a_2, \dots, a_n$. If a non-trivial solution exists (i.e., not all $a_i$ are zero), the vectors are linearly dependent.
    \end{itemize}
\end{itemize}

\section*{Simplified Explanation}

\textbf{Example 1: Linear Dependence in $\mathbb{R}^2$}
Given vectors $(1, 2)$, $(2, 1)$, and $(4, 5)$:
\begin{itemize}
  \item $(4, 5)$ is a linear combination of $(1, 2)$ and $(2, 1)$:
    \[
      2(1, 2) + (2, 1) = (4, 5).
    \]
  \item Rearranging:
    \[
      2(1, 2) + (2, 1) - (4, 5) = (0, 0).
    \]
  \item The zero vector is a linear combination of these vectors with non-zero coefficients, so they are linearly dependent.
\end{itemize}

\textbf{Example 2: Linear Dependence in $\mathbb{R}^3$}
Given vectors $(2, 1, 2)$, $(3, 0, 1)$, and $(10, -4, -2)$:
\begin{itemize}
  \item Equation:
    \[
      2(2, 1, 2) - 3(3, 0, 1) + \frac{1}{2}(10, -4, -2) = (0, 0, 0).
    \]
  \item Since non-zero coefficients exist, these vectors are linearly dependent.
\end{itemize}

\textbf{Example 3: Linear Independence}
Given $(0, 2, 1)$, $(2, 2, 0)$, and $(1, 2, 0)$:
\begin{itemize}
  \item Suppose:
    \[
      a(0, 2, 1) + b(2, 2, 0) + c(1, 2, 0) = (0, 0, 0).
    \]
  \item Solving gives $a = 0, b = 0, c = 0$. Hence, these vectors are linearly independent.
\end{itemize}

\section*{Conclusion}

In this lecture, we:
\begin{itemize}
  \item Defined linear dependence and provided the geometric intuition for $\mathbb{R}^2$ and $\mathbb{R}^3$.
  \item Demonstrated how to check for linear dependence using linear combinations.
  \item Highlighted that if a set of vectors is linearly dependent, any superset is also linearly dependent.
\end{itemize}

Linear dependence is a foundational concept for understanding the structure of vector spaces and solving systems of equations.

\end{document}
