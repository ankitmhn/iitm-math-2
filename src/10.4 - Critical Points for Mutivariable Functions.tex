\documentclass{article}
\usepackage{amsmath}
\usepackage{amssymb}
\usepackage{geometry}

\geometry{margin=1in}

\title{Lecture Summary: Critical Points for Multivariable Functions}
\author{}
\date{}

\begin{document}

\maketitle

\section*{Source: Week 11 Lec 04.pdf}

\section*{Key Points}

\begin{itemize}
  \item \textbf{Critical Points for Single-Variable Functions (Recap):}
    \begin{itemize}
      \item A point $a$ is a critical point of $f(x)$ if either:
        \[
          f'(a) = 0 \quad \text{or} \quad f'(a) \text{ does not exist}.
        \]
      \item Critical points include:
        \begin{itemize}
          \item Local maxima: $\frac{d^2f}{dx^2}(a) < 0$.
          \item Local minima: $\frac{d^2f}{dx^2}(a) > 0$.
          \item Saddle points: Neither maxima nor minima.
        \end{itemize}
    \end{itemize}

  \item \textbf{Extension to Multivariable Functions:}
    \begin{itemize}
      \item For $f: \mathbb{R}^n \to \mathbb{R}$, a point $\vec{a}$ is a critical point if:
        \[
          \nabla f(\vec{a}) = 0 \quad \text{or} \quad \nabla f(\vec{a}) \text{ does not exist}.
        \]
      \item Critical points can be:
        \begin{itemize}
          \item Local maxima: $f(\vec{x}) \leq f(\vec{a})$ in a neighborhood of $\vec{a}$.
          \item Local minima: $f(\vec{x}) \geq f(\vec{a})$ in a neighborhood of $\vec{a}$.
          \item Saddle points: Not local extrema.
        \end{itemize}
    \end{itemize}

  \item \textbf{Examples:}
    \begin{itemize}
      \item \textbf{Example 1:} $f(x, y) = \sin(xy)$:
        \begin{itemize}
          \item Gradient $\nabla f = (y\cos(xy), x\cos(xy))$.
          \item Setting $\nabla f = 0$ gives critical points:
            \[
              x = 0 \quad \text{or} \quad y = 0 \quad \text{or} \quad \cos(xy) = 0.
            \]
          \item $\cos(xy) = 0$ yields infinitely many critical points.
        \end{itemize}

      \item \textbf{Example 2:} $f(x, y) = x^2 + 6xy + 4y^2 - 2x - 4y$:
        \begin{itemize}
          \item Gradient $\nabla f = (2x + 6y + 2, 6x + 8y - 4)$.
          \item Solving $\nabla f = 0$ gives $x = 2$ and $y = -1$ as the critical point.
        \end{itemize}
    \end{itemize}

  \item \textbf{Saddle Points:}
    \begin{itemize}
      \item A critical point $\vec{a}$ is a saddle point if:
        \[
          \nabla f(\vec{a}) = 0 \quad \text{but} \quad \vec{a} \text{ is not a local extremum}.
        \]
      \item Example: $f(x, y) = x^2 - y^2$:
        \begin{itemize}
          \item At $(0, 0)$, $\nabla f = 0$, but $f$ is a maximum in some directions and a minimum in others.
        \end{itemize}
    \end{itemize}

  \item \textbf{Global Extrema:}
    \begin{itemize}
      \item A point $\vec{a}$ is a global maximum if:
        \[
          f(\vec{a}) \geq f(\vec{x}) \quad \text{for all } \vec{x} \in D.
        \]
      \item A point $\vec{a}$ is a global minimum if:
        \[
          f(\vec{a}) \leq f(\vec{x}) \quad \text{for all } \vec{x} \in D.
        \]
      \item For continuous functions on closed and bounded domains, global extrema always exist.
    \end{itemize}

  \item \textbf{Finding Global Extrema:}
    \begin{itemize}
      \item Check critical points within the domain.
      \item Evaluate $f$ on the boundary and reduce dimensions iteratively.
      \item Compare $f$ values at all critical points and boundaries to determine global extrema.
    \end{itemize}

  \item \textbf{Example of Global Extrema:}
    \begin{itemize}
      \item Function: $f(x, y) = x^3 + y^3 - 3x - 3y^2 + 1$ on a square domain.
      \item Steps:
        \begin{enumerate}
          \item Find critical points inside the domain: $(1, 0)$ and $(1, 2)$.
          \item Evaluate $f$ on the edges and corners of the square.
          \item Compare all values to determine:
            \[
              \text{Absolute Maximum: } f(2, 0) = 3, \quad \text{Absolute Minimum: } f(1, 2) = -5.
            \]
        \end{enumerate}
    \end{itemize}
\end{itemize}

\section*{Simplified Explanation}

\textbf{Critical Points:}
Points where the gradient is zero or undefined. These include potential maxima, minima, or saddle points.

\textbf{Example:}
For $f(x, y) = x^2 + 6xy + 4y^2 - 2x - 4y$, solving $\nabla f = 0$ yields $(2, -1)$ as a critical point.

\textbf{Global Extrema:}
To find the largest or smallest value of $f$ over a domain, check critical points and boundary values.

\section*{Conclusion}

In this lecture, we:
\begin{itemize}
  \item Defined critical points and explored their significance in multivariable functions.
  \item Distinguished between local extrema, saddle points, and global extrema.
  \item Demonstrated techniques to find global extrema on closed domains.
\end{itemize}

Critical points and extrema are fundamental concepts in optimization and analysis of multivariable functions.

\end{document}
