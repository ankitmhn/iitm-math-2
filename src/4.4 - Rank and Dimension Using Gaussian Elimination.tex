\documentclass{article}
\usepackage{amsmath}
\usepackage{amssymb}
\usepackage{geometry}

\geometry{margin=1in}

\title{Lecture Summary: Rank and Dimension Using Gaussian Elimination}
\author{}
\date{}

\begin{document}

\maketitle

\section*{Source: Lec35.pdf}

\section*{Key Points}

\begin{itemize}
  \item \textbf{Overview:}
    \begin{itemize}
      \item Gaussian elimination can be used to compute the rank and dimension of a vector space or subspace and to find a basis.
      \item This process is systematic and avoids ad-hoc methods, providing a clear algorithmic approach.
    \end{itemize}

  \item \textbf{Row Method:}
    \begin{itemize}
      \item \textbf{Steps:}
        \begin{enumerate}
          \item Arrange the spanning set vectors as rows of a matrix.
          \item Perform row reduction to bring the matrix to row echelon form.
          \item Count the number of non-zero rows:
            \begin{itemize}
              \item This is the dimension of the subspace.
              \item The non-zero rows form a basis for the subspace.
            \end{itemize}
        \end{enumerate}
      \item \textbf{Example: Subspace $W$ in $\mathbb{R}^3$ spanned by $\{(1, 0, 1), (-2, -3, 1), (3, 3, 0)\}$:}
        \begin{enumerate}
          \item Matrix:
            \[
              \begin{bmatrix}
                1 & 0 & 1 \\
                -2 & -3 & 1 \\
                3 & 3 & 0
              \end{bmatrix}.
            \]
          \item Row reduction yields:
            \[
              \begin{bmatrix}
                1 & 0 & 1 \\
                0 & 1 & -1 \\
                0 & 0 & 0
              \end{bmatrix}.
            \]
          \item Dimension = 2 (two non-zero rows).
          \item Basis: $\{(1, 0, 1), (0, 1, -1)\}$.
        \end{enumerate}
    \end{itemize}

  \item \textbf{Column Method:}
    \begin{itemize}
      \item \textbf{Steps:}
        \begin{enumerate}
          \item Arrange the spanning set vectors as columns of a matrix.
          \item Perform row reduction to row echelon form.
          \item Identify pivot columns (columns with leading ones).
          \item The original vectors corresponding to pivot columns form a basis.
        \end{enumerate}
      \item \textbf{Example: Subspace $W$ in $\mathbb{R}^3$ spanned by $\{(1, 0, 1), (-2, -3, 1), (3, 3, 0)\}$:}
        \begin{enumerate}
          \item Matrix:
            \[
              \begin{bmatrix}
                1 & -2 & 3 \\
                0 & -3 & 3 \\
                1 & 1 & 0
              \end{bmatrix}.
            \]
          \item Row reduction yields:
            \[
              \begin{bmatrix}
                1 & -2 & 3 \\
                0 & 1 & -1 \\
                0 & 0 & 0
              \end{bmatrix}.
            \]
          \item Pivot columns: 1 and 2.
          \item Basis: $\{(1, 0, 1), (-2, -3, 1)\}$.
        \end{enumerate}
    \end{itemize}

  \item \textbf{Comparison of Methods:}
    \begin{itemize}
      \item \textbf{Row Method:}
        \begin{itemize}
          \item Provides a basis directly from the rows of the row-reduced matrix.
          \item The basis vectors may not belong to the original spanning set.
        \end{itemize}
      \item \textbf{Column Method:}
        \begin{itemize}
          \item Provides a basis directly from the original spanning set.
          \item Useful when you need the basis to be composed of specific original vectors.
        \end{itemize}
    \end{itemize}
\end{itemize}

\section*{Simplified Explanation}

\textbf{Example 1: Row Method in $\mathbb{R}^4$}
Given vectors $\{(1, -2, 0, 4), (3, 1, 1, 0), (-1, -5, -1, 8), (3, 8, 2, -12)\}$:
\begin{itemize}
  \item Row reduction:
    \[
      \begin{bmatrix}
        1 & -2 & 0 & 4 \\
        0 & 1 & -1 & 2 \\
        0 & 0 & 0 & 0 \\
        0 & 0 & 0 & 0
      \end{bmatrix}.
    \]
  \item Dimension = 2.
  \item Basis: $\{(1, -2, 0, 4), (0, 1, -1, 2)\}$.
\end{itemize}

\textbf{Example 2: Column Method in $\mathbb{R}^4$}
Given the same vectors:
\begin{itemize}
  \item Column matrix:
    \[
      \begin{bmatrix}
        1 & 3 & -1 & 3 \\
        -2 & 1 & -5 & 8 \\
        0 & 1 & 2 & -12 \\
        4 & 0 & 8 & -12
      \end{bmatrix}.
    \]
  \item Pivot columns: 1 and 2.
  \item Basis: $\{(1, -2, 0, 4), (3, 1, 1, 0)\}$.
\end{itemize}

\section*{Conclusion}

In this lecture, we:
\begin{itemize}
  \item Explored two algorithmic methods for computing rank and dimension using Gaussian elimination.
  \item Compared the row method (basis from row-reduced matrix rows) and the column method (basis from original spanning set vectors).
  \item Highlighted examples in $\mathbb{R}^3$ and $\mathbb{R}^4$ to illustrate both methods.
\end{itemize}

Gaussian elimination provides a versatile approach for basis and dimension computation, adaptable to various applications.

\end{document}
