\documentclass{article}
\usepackage{amsmath}
\usepackage{amssymb}
\usepackage{geometry}

\geometry{margin=1in}

\title{Lecture Summary: Null Space, Nullity, and Basis}
\author{}
\date{}

\begin{document}

\maketitle

\section*{Source: Lec36.pdf}

\section*{Key Points}

\begin{itemize}
  \item \textbf{Definition of Null Space:}
    \begin{itemize}
      \item Let $A$ be an $m \times n$ matrix. The \textbf{null space} of $A$ is defined as:
        \[
          \text{Null}(A) = \{x \in \mathbb{R}^n \mid Ax = 0\}.
        \]
      \item It represents the solution space of the homogeneous system $Ax = 0$.
      \item Null space is a subspace of $\mathbb{R}^n$.
    \end{itemize}

  \item \textbf{Nullity:}
    \begin{itemize}
      \item The \textbf{nullity} of $A$ is the dimension of the null space of $A$.
      \item Denoted as $\text{nullity}(A)$, it measures the number of independent solutions to $Ax = 0$.
    \end{itemize}

  \item \textbf{Subspace Verification:}
    \begin{itemize}
      \item To confirm that the null space is a subspace:
        \begin{enumerate}
          \item If $x, y \in \text{Null}(A)$, then $x + y \in \text{Null}(A)$ (closure under addition).
          \item If $x \in \text{Null}(A)$ and $\lambda \in \mathbb{R}$, then $\lambda x \in \text{Null}(A)$ (closure under scalar multiplication).
        \end{enumerate}
    \end{itemize}

  \item \textbf{Finding the Basis for the Null Space:}
    \begin{itemize}
      \item Use Gaussian elimination to row reduce $A$.
      \item Identify independent and dependent variables:
        \begin{enumerate}
          \item Columns with leading 1s correspond to dependent variables.
          \item Other columns correspond to independent variables.
        \end{enumerate}
      \item Assign parameters (e.g., $t_1, t_2$) to independent variables.
      \item Solve for dependent variables in terms of the independent variables.
      \item Substitute values for one parameter at a time (e.g., $t_i = 1$, all others = 0) to construct basis vectors.
    \end{itemize}

  \item \textbf{Rank-Nullity Theorem:}
    \begin{itemize}
      \item The rank-nullity theorem states:
        \[
          \text{rank}(A) + \text{nullity}(A) = n,
        \]
        where $n$ is the number of columns of $A$.
    \end{itemize}
\end{itemize}

\section*{Simplified Explanation}

\textbf{Example: Null Space and Basis in $\mathbb{R}^3$}
Given:
\[
  A =
  \begin{bmatrix}
    1 & 1 & 1 \\
    2 & 2 & 2 \\
    3 & 3 & 3
  \end{bmatrix}.
\]
Steps:
\begin{enumerate}
  \item Augmented matrix:
    \[
      [A | 0] =
      \begin{bmatrix}
        1 & 1 & 1 & 0 \\
        2 & 2 & 2 & 0 \\
        3 & 3 & 3 & 0
      \end{bmatrix}.
    \]
  \item Row reduce:
    \[
      \begin{bmatrix}
        1 & 1 & 1 & 0 \\
        0 & 0 & 0 & 0 \\
        0 & 0 & 0 & 0
      \end{bmatrix}.
    \]
  \item Identify variables:
    \begin{itemize}
      \item $x_1$ is dependent.
      \item $x_2, x_3$ are independent.
    \end{itemize}
  \item Solve: $x_1 = -x_2 - x_3$.
  \item Basis for null space:
    \[
      \{(-1, 1, 0), (-1, 0, 1)\}.
    \]
\end{enumerate}

\textbf{Example 2: Applying the Rank-Nullity Theorem}
\begin{itemize}
  \item $A$ has 3 columns.
  \item Row-reduced matrix has 1 non-zero row $\Rightarrow \text{rank}(A) = 1$.
  \item Nullity:
    \[
      \text{nullity}(A) = 3 - 1 = 2.
    \]
\end{itemize}

\section*{Conclusion}

In this lecture, we:
\begin{itemize}
  \item Defined the null space and nullity of a matrix.
  \item Demonstrated how to find the basis and nullity using Gaussian elimination.
  \item Applied the rank-nullity theorem to relate rank and nullity.
\end{itemize}

Null space analysis is critical in solving systems of equations and understanding the structure of matrices in linear algebra.

\end{document}
