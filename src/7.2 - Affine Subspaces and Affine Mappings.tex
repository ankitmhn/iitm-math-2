\documentclass{article}
\usepackage{amsmath}
\usepackage{amssymb}
\usepackage{geometry}

\geometry{margin=1in}

\title{Lecture Summary: Affine Subspaces and Affine Mappings}
\author{}
\date{}

\begin{document}

\maketitle

\section*{Source: Lec45.pdf}

\section*{Key Points}

\begin{itemize}
  \item \textbf{Definition of Affine Subspace:}
    \begin{itemize}
      \item An affine subspace $L$ of a vector space $V$ is a subset of $V$ defined as:
        \[
          L = v + U,
        \]
        where:
        \begin{itemize}
          \item $v \in V$ is a fixed vector.
          \item $U$ is a vector subspace of $V$.
        \end{itemize}
      \item $L$ is formed by translating the subspace $U$ by the vector $v$.
    \end{itemize}

  \item \textbf{Dimension of Affine Subspace:}
    \begin{itemize}
      \item If $U$ is $n$-dimensional, the corresponding affine subspace $L$ is also considered $n$-dimensional.
      \item The subspace $U$ associated with $L$ is unique.
    \end{itemize}

  \item \textbf{Affine Subspaces in $\mathbb{R}^n$:}
    \begin{itemize}
      \item In $\mathbb{R}^2$, affine subspaces include:
        \begin{itemize}
          \item Points (shifting the origin by a vector).
          \item Lines (translating a line passing through the origin).
          \item The entire plane $\mathbb{R}^2$ (a subspace translated by 0).
        \end{itemize}
      \item In $\mathbb{R}^3$, affine subspaces include:
        \begin{itemize}
          \item Points.
          \item Lines.
          \item Planes.
          \item The entire space $\mathbb{R}^3$.
        \end{itemize}
    \end{itemize}

  \item \textbf{Affine Subspace as Solution Sets:}
    \begin{itemize}
      \item For a linear system $Ax = b$, the solution set $L$ is:
        \[
          L = v + n(A),
        \]
        where:
        \begin{itemize}
          \item $n(A)$ is the null space of $A$.
          \item $v$ is any particular solution to $Ax = b$.
        \end{itemize}
      \item If $b = 0$, the system is homogeneous, and $L = n(A)$ is a subspace.
      \item If $b \neq 0$ and lies in the column space of $A$, the solution set $L$ is an affine subspace.
    \end{itemize}

  \item \textbf{Affine Mappings:}
    \begin{itemize}
      \item Let $L$ and $L'$ be affine subspaces of $V$ and $W$, respectively.
      \item A function $f: L \to L'$ is an \textbf{affine mapping} if:
        \[
          f(v + u) = f(v) + T(u),
        \]
        where:
        \begin{itemize}
          \item $T: U \to U'$ is a linear transformation.
          \item $v \in L$, $u \in U$, and $f(v) \in L'$.
        \end{itemize}
      \item Every affine mapping can be decomposed as a translation followed by a linear transformation.
    \end{itemize}

  \item \textbf{Example: Affine Mapping in $\mathbb{R}^3$:}
    \begin{itemize}
      \item Define $f(x, y, z) = (2x + 3y + 2, 4x - 5y + 3)$.
      \item This is not a linear transformation since $f(0, 0, 0) \neq (0, 0)$.
      \item However, $f$ can be written as:
        \[
          f(x, y, z) = (2, 3) + (2x + 3y, 4x - 5y),
        \]
        where the second term is a linear transformation.
      \item Therefore, $f$ is an affine mapping.
    \end{itemize}
\end{itemize}

\section*{Simplified Explanation}

\textbf{Affine Subspace}
Affine subspaces generalize vector subspaces by allowing translation:
\begin{itemize}
  \item Example in $\mathbb{R}^2$: A line not passing through the origin is an affine subspace obtained by translating a line passing through the origin.
  \item Solution sets of non-homogeneous systems $Ax = b$ are affine subspaces when $b$ lies in the column space of $A$.
\end{itemize}

\textbf{Affine Mapping}
Affine mappings allow linear transformations with a translation:
\begin{itemize}
  \item Example: $f(x, y) = (2x + 1, y - 1)$ translates by $(1, -1)$ and scales $(x, y)$ by linear rules.
\end{itemize}

\section*{Conclusion}

In this lecture, we:
\begin{itemize}
  \item Introduced affine subspaces as translations of vector subspaces.
  \item Showed how solution sets of linear equations form affine subspaces.
  \item Defined affine mappings and demonstrated their decomposition into translations and linear transformations.
\end{itemize}

Affine subspaces and mappings extend linear algebra concepts, particularly in solving non-homogeneous systems and transformations in affine geometry.

\end{document}
