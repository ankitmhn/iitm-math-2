\documentclass{article}
\usepackage{amsmath}
\usepackage{amssymb}
\usepackage{geometry}

\geometry{margin=1in}

\title{Lecture Summary: The Gram-Schmidt Process}
\author{}
\date{}

\begin{document}

\maketitle

\section*{Source: The Gram-Schmidt process.pdf}

\section*{Key Points}

\begin{itemize}
  \item \textbf{Definition of Gram-Schmidt Process:}
    \begin{itemize}
      \item The Gram-Schmidt process converts any basis $\{x_1, x_2, \dots, x_n\}$ of an inner product space into an orthonormal basis $\{w_1, w_2, \dots, w_n\}$.
      \item An orthonormal basis is a set of mutually orthogonal vectors, each with norm 1:
        \[
          \langle w_i, w_j \rangle = 0 \quad \text{for } i \neq j, \quad \|w_i\| = 1.
        \]
    \end{itemize}

  \item \textbf{Procedure for Gram-Schmidt Process:}
    \begin{enumerate}
      \item Start with a basis $\{x_1, x_2, \dots, x_n\}$.
      \item Define $v_1 = x_1$ and $w_1 = \frac{v_1}{\|v_1\|}$.
      \item For $i \geq 2$, define:
        \[
          v_i = x_i - \sum_{j=1}^{i-1} \langle x_i, w_j \rangle w_j, \quad w_i = \frac{v_i}{\|v_i\|}.
        \]
      \item The result is an orthonormal basis $\{w_1, w_2, \dots, w_n\}$.
    \end{enumerate}

  \item \textbf{Key Concepts:}
    \begin{itemize}
      \item The Gram-Schmidt process relies on projections to iteratively remove components of $x_i$ in the direction of the previous orthonormal vectors.
      \item Each $v_i$ is orthogonal to all $v_j$ for $j < i$, ensuring orthogonality.
    \end{itemize}

  \item \textbf{Examples:}
    \begin{itemize}
      \item Example in $\mathbb{R}^3$: Starting basis $\{(1, 2, 2), (-1, 0, 2), (0, 0, 1)\}$:
        \begin{itemize}
          \item Step 1: $v_1 = (1, 2, 2)$, $w_1 = \frac{1}{3}(1, 2, 2)$.
          \item Step 2: $v_2 = (-1, 0, 2) - \langle (-1, 0, 2), w_1 \rangle w_1$.
            \[
              \langle (-1, 0, 2), w_1 \rangle = \frac{1}{3}(-1 \cdot 1 + 0 \cdot 2 + 2 \cdot 2) = \frac{3}{9}.
            \]
            \[
              v_2 = (-1, 0, 2) - \frac{1}{3}(1, 2, 2) = (-\frac{4}{3}, -\frac{2}{3}, \frac{4}{3}), \quad w_2 = \frac{1}{3}(-4, -2, 4).
            \]
          \item Step 3: $v_3 = (0, 0, 1) - \langle (0, 0, 1), w_1 \rangle w_1 - \langle (0, 0, 1), w_2 \rangle w_2$.
            \[
              w_3 = \frac{1}{3}(2, -2, 1).
            \]
          \item Orthonormal basis:
            \[
              \left\{\frac{1}{3}(1, 2, 2), \frac{1}{3}(-4, -2, 4), \frac{1}{3}(2, -2, 1)\right\}.
            \]
        \end{itemize}
    \end{itemize}

  \item \textbf{Applications of Gram-Schmidt Process:}
    \begin{itemize}
      \item Converting any basis into an orthonormal basis for simplified computations in linear algebra.
      \item Facilitating projections and decompositions, such as in QR decomposition of matrices.
      \item Useful in functional analysis, signal processing, and data science.
    \end{itemize}
\end{itemize}

\section*{Simplified Explanation}

\textbf{Gram-Schmidt Process:}
A step-by-step method to convert any basis into an orthonormal basis by iteratively removing components along previous directions.

\textbf{Example in $\mathbb{R}^3$:}
Start with $(1, 2, 2)$, $(-1, 0, 2)$, $(0, 0, 1)$:
\begin{itemize}
  \item Normalize $(1, 2, 2)$ to get the first orthonormal vector.
  \item Subtract projections to make $(-1, 0, 2)$ orthogonal to $(1, 2, 2)$, then normalize.
  \item Repeat for $(0, 0, 1)$ to make it orthogonal to both previous vectors, then normalize.
\end{itemize}

\textbf{Applications:}
The Gram-Schmidt process is critical in transforming vector spaces, enabling simplified calculations, particularly with projections and decompositions.

\section*{Conclusion}

In this lecture, we:
\begin{itemize}
  \item Defined the Gram-Schmidt process.
  \item Demonstrated its use in generating orthonormal bases from arbitrary bases.
  \item Highlighted applications in linear algebra and computational mathematics.
\end{itemize}

The Gram-Schmidt process is a foundational tool in linear algebra, widely used in both theoretical and applied contexts.

\end{document}
