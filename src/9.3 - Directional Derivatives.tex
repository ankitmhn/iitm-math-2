\documentclass{article}
\usepackage{amsmath}
\usepackage{amssymb}
\usepackage{geometry}

\geometry{margin=1in}

\title{Lecture Summary: Directional Derivatives}
\author{}
\date{}

\begin{document}

\maketitle

\section*{Source: Directional Derivatives.pdf}

\section*{Key Points}

\begin{itemize}
  \item \textbf{Definition of Directional Derivative:}
    \begin{itemize}
      \item The directional derivative of a scalar function $f(x_1, x_2, \dots, x_n)$ at a point $\vec{a}$ in the direction of a unit vector $\vec{u}$ is:
        \[
          D_{\vec{u}} f(\vec{a}) = \lim_{h \to 0} \frac{f(\vec{a} + h \vec{u}) - f(\vec{a})}{h}.
        \]
      \item It measures the rate of change of $f$ in the direction of $\vec{u}$.
    \end{itemize}

  \item \textbf{Relation to Partial Derivatives:}
    \begin{itemize}
      \item If $\vec{u} = \vec{e}_i$ (a standard unit vector), the directional derivative reduces to the partial derivative:
        \[
          D_{\vec{e}_i} f(\vec{a}) = \frac{\partial f}{\partial x_i}(\vec{a}).
        \]
      \item Directional derivatives generalize partial derivatives to arbitrary directions.
    \end{itemize}

  \item \textbf{Computing Directional Derivatives:}
    \begin{itemize}
      \item If $\vec{u} = (u_1, u_2, \dots, u_n)$ and $\vec{a} = (a_1, a_2, \dots, a_n)$, then:
        \[
          D_{\vec{u}} f(\vec{a}) = \sum_{i=1}^n u_i \frac{\partial f}{\partial x_i}(\vec{a}).
        \]
      \item This formula shows that directional derivatives are a weighted sum of partial derivatives.
    \end{itemize}

  \item \textbf{Examples:}
    \begin{itemize}
      \item Example 1: For $f(x, y) = x + y$ at $(0, 0)$ in the direction of $\vec{u} = \frac{1}{\sqrt{2}}(1, 1)$:
        \[
          D_{\vec{u}} f(0, 0) = \frac{1}{\sqrt{2}} \cdot 1 + \frac{1}{\sqrt{2}} \cdot 1 = \sqrt{2}.
        \]
      \item Example 2: For $f(x, y, z) = xy + yz + zx$ at $(1, 2, 3)$ in the direction of $\vec{u} = \frac{1}{5}(4, 3, 0)$:
        \[
          D_{\vec{u}} f(1, 2, 3) = \frac{4}{5} \cdot 5 + \frac{3}{5} \cdot 4 = \frac{32}{5}.
        \]
    \end{itemize}

  \item \textbf{Properties of Directional Derivatives:}
    \begin{itemize}
      \item \textbf{Linearity:}
        \[
          D_{\vec{u}} (c f + g)(\vec{a}) = c D_{\vec{u}} f(\vec{a}) + D_{\vec{u}} g(\vec{a}).
        \]
      \item \textbf{Product Rule:}
        \[
          D_{\vec{u}} (f \cdot g)(\vec{a}) = f(\vec{a}) D_{\vec{u}} g(\vec{a}) + g(\vec{a}) D_{\vec{u}} f(\vec{a}).
        \]
      \item \textbf{Quotient Rule:}
        \[
          D_{\vec{u}} \left(\frac{f}{g}\right)(\vec{a}) = \frac{g(\vec{a}) D_{\vec{u}} f(\vec{a}) - f(\vec{a}) D_{\vec{u}} g(\vec{a})}{g(\vec{a})^2}.
        \]
    \end{itemize}

  \item \textbf{Geometric Interpretation:}
    \begin{itemize}
      \item The directional derivative is the slope of the graph of $f$ along the line in the direction of $\vec{u}$.
      \item It captures the instantaneous rate of change of $f$ at a point in any direction.
    \end{itemize}
\end{itemize}

\section*{Simplified Explanation}

\textbf{What is a Directional Derivative?}
It measures how a function changes as we move in a specific direction, generalizing the concept of partial derivatives.

\textbf{How to Compute It?}
Use the formula:
\[
  D_{\vec{u}} f(\vec{a}) = \sum u_i \frac{\partial f}{\partial x_i}(\vec{a}),
\]
where $u_i$ are components of the unit vector $\vec{u}$.

\textbf{Examples:}
For $f(x, y) = x + y$ at $(0, 0)$, moving diagonally at $45^\circ$ (direction $\vec{u} = (1/\sqrt{2}, 1/\sqrt{2})$), the rate of change is $\sqrt{2}$.

\section*{Conclusion}

In this lecture, we:
\begin{itemize}
  \item Defined directional derivatives as a generalization of partial derivatives.
  \item Demonstrated computation techniques using both limits and formulas.
  \item Highlighted their geometric and practical significance.
\end{itemize}

Directional derivatives are fundamental tools in multivariable calculus, crucial for understanding gradients and optimization.

\end{document}
