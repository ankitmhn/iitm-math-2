\documentclass{article}
\usepackage{amsmath}
\usepackage{amssymb}
\usepackage{geometry}

\geometry{margin=1in}

\title{Lecture Summary: Echelon Form}
\author{}
\date{}

\begin{document}

\maketitle

\section*{Source: Lec24.pdf}

\section*{Key Points}

\begin{itemize}
  \item \textbf{Definition of Echelon Form:}
    \begin{itemize}
      \item A matrix is in \textbf{Row Echelon Form (REF)} if:
        \begin{enumerate}
          \item The first non-zero element in each row (called the \textit{leading entry}) is 1.
          \item Each leading entry is to the right of the leading entry in the row above.
          \item Rows with all zero elements are at the bottom.
        \end{enumerate}
      \item A matrix is in \textbf{Reduced Row Echelon Form (RREF)} if, in addition to being in REF:
        \begin{enumerate}
          \item Each column containing a leading 1 has all other entries in that column as 0.
        \end{enumerate}
    \end{itemize}

  \item \textbf{Properties of Echelon Forms:}
    \begin{itemize}
      \item Solutions to a system of linear equations can be easily obtained when the coefficient matrix is in echelon form.
      \item Independent variables are those corresponding to columns without leading 1s.
      \item Dependent variables correspond to columns containing leading 1s.
    \end{itemize}

  \item \textbf{Procedure to Solve $Ax = b$:}
    \begin{itemize}
      \item Convert the coefficient matrix $A$ to RREF.
      \item Identify independent and dependent variables.
      \item Assign arbitrary values to independent variables and compute dependent variables using the equations.
    \end{itemize}

  \item \textbf{Homogeneous Systems:}
    \begin{itemize}
      \item For $Ax = 0$, if $A$ is in RREF:
        \begin{enumerate}
          \item If there are no zero rows, the system has a unique solution ($x = 0$).
          \item If there are zero rows, the system has infinitely many solutions, parameterized by the independent variables.
        \end{enumerate}
    \end{itemize}

  \item \textbf{Inconsistent Systems:}
    \begin{itemize}
      \item If a row in $A$ is all zeros, but the corresponding entry in $b$ is non-zero, the system has no solution.
    \end{itemize}
\end{itemize}

\section*{Simplified Explanation}

\textbf{Example: REF and RREF}
For $A =
\begin{bmatrix}
  1 & 0 & 2 \\
  0 & 1 & 3
\end{bmatrix}$:
\begin{itemize}
  \item REF:
    \begin{enumerate}
      \item The first non-zero element in each row is 1.
      \item The leading 1 in the second row is to the right of the leading 1 in the first row.
      \item There are no zero rows.
    \end{enumerate}
  \item RREF:
    \begin{enumerate}
      \item Each column containing a leading 1 has all other entries as 0.
    \end{enumerate}
\end{itemize}

\textbf{Example: Solving $Ax = b$}
Given $Ax = b$ with $A$ in RREF:
\[
  A =
  \begin{bmatrix}
    1 & 0 & 2 \\
    0 & 1 & 3
  \end{bmatrix},
  \quad
  b =
  \begin{bmatrix}
    b_1 \\
    b_2
  \end{bmatrix}.
\]
Equations:
\[
  x_1 + 2x_3 = b_1, \quad x_2 + 3x_3 = b_2.
\]
\begin{itemize}
  \item Assign $x_3 = c$ (independent variable).
  \item Solve for dependent variables:
    \[
      x_1 = b_1 - 2c, \quad x_2 = b_2 - 3c.
    \]
  \item General solution:
    \[
      x =
      \begin{bmatrix}
        b_1 - 2c \\
        b_2 - 3c \\
        c
      \end{bmatrix}.
    \]
\end{itemize}

\textbf{Inconsistent Case:}
If a row in $A$ is all zeros but the corresponding $b_i \neq 0$, then the system is inconsistent (no solution).

\section*{Conclusion}

In this lecture, we:
\begin{itemize}
  \item Defined Row Echelon Form (REF) and Reduced Row Echelon Form (RREF).
  \item Showed how to use RREF to solve systems of linear equations efficiently.
  \item Analyzed solutions for consistent, inconsistent, and homogeneous systems.
\end{itemize}

This approach provides an algorithmic method to find all solutions to $Ax = b$, leveraging the structure of echelon forms.

\end{document}
