\documentclass{article}
\usepackage{amsmath}
\usepackage{amssymb}
\usepackage{geometry}

\geometry{margin=1in}

\title{Lecture Summary: Introduction to Linear Mappings – Part 1}
\author{}
\date{}

\begin{document}

\maketitle

\section*{Source: Lec38.pdf}

\section*{Key Points}

\begin{itemize}
  \item \textbf{Introduction to Linear Mappings:}
    \begin{itemize}
      \item A linear mapping represents a function that transforms inputs to outputs while preserving the properties of linear combinations.
      \item Example: Computing the cost of goods in a grocery shop using quantities and prices as inputs.
    \end{itemize}

  \item \textbf{Example: Grocery Shop Cost Function:}
    \begin{itemize}
      \item Prices at Shop A:
        \[
          \text{Rice: } 45 \, \text{Rs/kg}, \, \text{Dal: } 125 \, \text{Rs/kg}, \, \text{Oil: } 150 \, \text{Rs/liter}.
        \]
      \item Cost for given quantities $x_1$ (rice), $x_2$ (dal), $x_3$ (oil):
        \[
          c_A(x_1, x_2, x_3) = 45x_1 + 125x_2 + 150x_3.
        \]
      \item This is a linear combination of $x_1, x_2, x_3$ with coefficients $45, 125, 150$.
      \item Example computations:
        \[
          c_A(1, 2, 1) = 45 \cdot 1 + 125 \cdot 2 + 150 \cdot 1 = 445 \, \text{Rs}.
        \]
        \[
          c_A(2, 1, 2) = 45 \cdot 2 + 125 \cdot 1 + 150 \cdot 2 = 515 \, \text{Rs}.
        \]
    \end{itemize}

  \item \textbf{Generalization:}
    \begin{itemize}
      \item A cost function $c_A: \mathbb{R}^3 \to \mathbb{R}$ is defined as:
        \[
          c_A(x_1, x_2, x_3) = 45x_1 + 125x_2 + 150x_3.
        \]
      \item Inputs are represented as vectors:
        \[
          \begin{bmatrix}
            x_1 \\
            x_2 \\
            x_3
          \end{bmatrix},
        \]
        and the cost function is equivalent to a matrix-vector multiplication:
        \[
          c_A(x) =
          \begin{bmatrix}
            45 & 125 & 150
          \end{bmatrix}
          \begin{bmatrix}
            x_1 \\
            x_2 \\
            x_3
          \end{bmatrix}.
        \]
    \end{itemize}

  \item \textbf{Properties of Linear Mappings:}
    \begin{itemize}
      \item Linearity ensures that combinations of inputs can be simplified:
        \[
          c_A(a \cdot x + b \cdot y) = a \cdot c_A(x) + b \cdot c_A(y).
        \]
      \item This property allows for simplified computations in real-world scenarios.
    \end{itemize}

  \item \textbf{Case Study: Caterer’s Orders:}
    \begin{itemize}
      \item Office 1 requires 20 kg rice, 10 kg dal, 4 liters oil:
        \[
          c_A(20, 10, 4) = 2750 \, \text{Rs}.
        \]
      \item Office 2 requires 30 kg rice, 12 kg dal, 2 liters oil:
        \[
          c_A(30, 12, 2) = 3150 \, \text{Rs}.
        \]
      \item For combined orders (Wednesday):
        \[
          c_A\left(\frac{1}{2}(20, 10, 4) + \frac{5}{4}(30, 12, 2)\right) = 5312.5 \, \text{Rs}.
        \]
      \item Linearity allows splitting computations into contributions from Monday and Tuesday orders.
    \end{itemize}

  \item \textbf{Key Observations:}
    \begin{itemize}
      \item A linear function can be interpreted as a combination of contributions weighted by inputs.
      \item Linearity simplifies real-world calculations like those in commerce and logistics.
    \end{itemize}
\end{itemize}

\section*{Simplified Explanation}

\textbf{Example: Linear Cost Function}
Prices: Rice = 45 Rs/kg, Dal = 125 Rs/kg, Oil = 150 Rs/liter.
\begin{itemize}
  \item For quantities $x_1 = 2$ (rice), $x_2 = 1$ (dal), $x_3 = 2$ (oil):
    \[
      c_A(x_1, x_2, x_3) = 45 \cdot 2 + 125 \cdot 1 + 150 \cdot 2 = 515 \, \text{Rs}.
    \]
  \item Combined inputs maintain the linearity:
    \[
      c_A\left(a \cdot x_1 + b \cdot x_2\right) = a \cdot c_A(x_1) + b \cdot c_A(x_2).
    \]
\end{itemize}

\section*{Conclusion}

In this lecture, we:
\begin{itemize}
  \item Introduced linear mappings as functions preserving linear combinations.
  \item Illustrated linearity through examples like cost functions and catering orders.
  \item Highlighted the computational advantages of linearity in real-world scenarios.
\end{itemize}

This sets the foundation for exploring more advanced applications of linear mappings in data science and mathematics.

\end{document}
