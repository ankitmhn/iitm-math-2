\documentclass{article}
\usepackage{amsmath}
\usepackage{amssymb}
\usepackage{geometry}

\geometry{margin=1in}

\title{Lecture Summary: Image and Kernel of Linear Transformations}
\author{}
\date{}

\begin{document}

\maketitle

\section*{Source: Lec42.pdf}

\section*{Key Points}

\begin{itemize}
  \item \textbf{Definition of Kernel:}
    \begin{itemize}
      \item For a linear transformation $f: V \to W$, the \textbf{kernel} of $f$ is defined as:
        \[
          \ker(f) = \{v \in V \mid f(v) = 0\}.
        \]
      \item The kernel is a subspace of $V$:
        \begin{itemize}
          \item Closure under addition: If $v_1, v_2 \in \ker(f)$, then $v_1 + v_2 \in \ker(f)$.
          \item Closure under scalar multiplication: If $v \in \ker(f)$ and $\alpha \in \mathbb{R}$, then $\alpha v \in \ker(f)$.
        \end{itemize}
    \end{itemize}

  \item \textbf{Definition of Image:}
    \begin{itemize}
      \item For $f: V \to W$, the \textbf{image} of $f$ (also called the range) is:
        \[
          \operatorname{Im}(f) = \{w \in W \mid \exists v \in V, \, w = f(v)\}.
        \]
      \item The image is a subspace of $W$:
        \begin{itemize}
          \item Closure under addition: If $w_1, w_2 \in \operatorname{Im}(f)$, then $w_1 + w_2 \in \operatorname{Im}(f)$.
          \item Closure under scalar multiplication: If $w \in \operatorname{Im}(f)$ and $\alpha \in \mathbb{R}$, then $\alpha w \in \operatorname{Im}(f)$.
        \end{itemize}
    \end{itemize}

  \item \textbf{Kernel and Injectivity:}
    \begin{itemize}
      \item A linear transformation $f$ is injective if and only if $\ker(f) = \{0\}$ (the zero subspace).
    \end{itemize}

  \item \textbf{Image and Surjectivity:}
    \begin{itemize}
      \item A linear transformation $f$ is surjective if and only if $\operatorname{Im}(f) = W$.
    \end{itemize}

  \item \textbf{Relation to Matrices:}
    \begin{itemize}
      \item For $f: V \to W$, let $\beta = \{v_1, \dots, v_n\}$ and $\gamma = \{w_1, \dots, w_m\}$ be ordered bases for $V$ and $W$, respectively.
      \item The corresponding matrix $A$ is defined by expressing $f(v_j)$ as a linear combination of $\{w_1, \dots, w_m\}$:
        \[
          f(v_j) = \sum_{i=1}^m a_{ij} w_i.
        \]
      \item The $j$th column of $A$ consists of the coefficients $a_{ij}$.
    \end{itemize}

  \item \textbf{Kernel and Null Space:}
    \begin{itemize}
      \item $\ker(f)$ corresponds to the null space of $A$, which is:
        \[
          \text{Null}(A) = \{x \in \mathbb{R}^n \mid Ax = 0\}.
        \]
      \item The kernel is isomorphic to the null space of the associated matrix $A$.
    \end{itemize}

  \item \textbf{Image and Column Space:}
    \begin{itemize}
      \item $\operatorname{Im}(f)$ corresponds to the column space of $A$, which is the span of the columns of $A$.
      \item Finding a basis for $\operatorname{Im}(f)$ involves identifying pivot columns of $A$ in row-reduced form.
    \end{itemize}

  \item \textbf{Summary of Isomorphisms:}
    \begin{itemize}
      \item $\ker(f)$ is isomorphic to the null space of $A$.
      \item $\operatorname{Im}(f)$ is isomorphic to the column space of $A$.
    \end{itemize}
\end{itemize}

\section*{Simplified Explanation}

\textbf{Example 1: Linear Transformation on $\mathbb{R}^2$}
Let $f: \mathbb{R}^2 \to \mathbb{R}^2$ be defined as:
\[
  f(x, y) = (2x, y).
\]
\begin{itemize}
  \item Kernel: Solve $f(x, y) = (0, 0)$:
    \[
      (2x, y) = (0, 0) \implies x = 0, y = 0.
    \]
  \item $\ker(f) = \{(0, 0)\}$ (the zero subspace).
  \item Image: Check if every $(u, v) \in \mathbb{R}^2$ can be written as:
    \[
      (2x, y) = (u, v).
    \]
  \item Solve:
    \[
      x = \frac{u}{2}, \, y = v.
    \]
  \item $\operatorname{Im}(f) = \mathbb{R}^2$ (entire space).
\end{itemize}

\textbf{Example 2: Linear Transformation on $\mathbb{R}^2$}
Let $f: \mathbb{R}^2 \to \mathbb{R}^2$ be defined as:
\[
  f(x, y) = (2x, 0).
\]
\begin{itemize}
  \item Kernel: Solve $f(x, y) = (0, 0)$:
    \[
      (2x, 0) = (0, 0) \implies x = 0.
    \]
  \item $\ker(f) = \{(0, y) \mid y \in \mathbb{R}\}$ (the $y$-axis).
  \item Image: Check if every $(u, v) \in \mathbb{R}^2$ can be written as:
    \[
      (2x, 0) = (u, v).
    \]
  \item $v = 0$, $u = 2x \implies x = \frac{u}{2}$.
  \item $\operatorname{Im}(f) = \{(u, 0) \mid u \in \mathbb{R}\}$ (the $x$-axis).
\end{itemize}

\section*{Conclusion}

In this lecture, we:
\begin{itemize}
  \item Defined the kernel and image of linear transformations.
  \item Connected kernels to null spaces and images to column spaces.
  \item Established isomorphisms between these subspaces and their matrix representations.
  \item Demonstrated examples to compute kernel and image using matrices.
\end{itemize}

This lays the foundation for understanding the structure of linear transformations and their interplay with matrices.

\end{document}
