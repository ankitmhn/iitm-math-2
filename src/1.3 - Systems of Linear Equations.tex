\documentclass{article}
\usepackage{amsmath}
\usepackage{amssymb}
\usepackage{geometry}

\geometry{margin=1in}

\title{Lecture Summary: Systems of Linear Equations}
\author{}
\date{}

\begin{document}

\maketitle

\section*{Source: Lec 18.pdf}

\section*{Key Points}

\begin{itemize}
  \item \textbf{Definition:}
    \begin{itemize}
      \item A system of linear equations consists of multiple linear equations with the same set of unknowns.
      \item A linear equation is of the form:
        \[
          a_1x_1 + a_2x_2 + \cdots + a_nx_n = b,
        \]
        where $x_1, x_2, \ldots, x_n$ are unknowns, $a_1, a_2, \ldots, a_n$ are coefficients, and $b$ is a constant.
    \end{itemize}

  \item \textbf{Matrix Representation:}
    \begin{itemize}
      \item Any system of linear equations can be represented in matrix form as:
        \[
          A \cdot x = b,
        \]
        where:
        \[
          A =
          \begin{bmatrix}
            a_{11} & a_{12} & \cdots & a_{1n} \\
            a_{21} & a_{22} & \cdots & a_{2n} \\
            \vdots & \vdots & \ddots & \vdots \\
            a_{m1} & a_{m2} & \cdots & a_{mn}
          \end{bmatrix},
          \quad
          x =
          \begin{bmatrix}
            x_1 \\
            x_2 \\
            \vdots \\
            x_n
          \end{bmatrix},
          \quad
          b =
          \begin{bmatrix}
            b_1 \\
            b_2 \\
            \vdots \\
            b_m
          \end{bmatrix}.
        \]
    \end{itemize}

  \item \textbf{Solutions to Systems:}
    \begin{itemize}
      \item A system of linear equations can have:
        \begin{enumerate}
          \item No solution (inconsistent system),
          \item A unique solution,
          \item Infinitely many solutions.
        \end{enumerate}
    \end{itemize}
\end{itemize}

\section*{Simplified Explanation}

\textbf{Example 1: Unique Solution}
Buyer A buys 2 kg of rice and 1 kg of dal, while buyer B buys 3 kg of rice and 1 kg of dal. They pay Rs. 215 and Rs. 260, respectively.
The system of linear equations is:
\[
  2x + y = 215, \quad 3x + y = 260,
\]
where $x$ is the price of rice per kg and $y$ is the price of dal per kg.
Solving the equations:
\[
  x = 45, \quad y = 125.
\]

\textbf{Example 2: No Solution}
If buyer A pays Rs. 215 and buyer B pays Rs. 400 for the same quantities, the equations become:
\[
  2x + y = 215, \quad 4x + 2y = 400.
\]
Doubling the first equation gives:
\[
  4x + 2y = 430,
\]
which contradicts the second equation ($430 \neq 400$). Hence, no solution exists.

\textbf{Example 3: Infinitely Many Solutions}
If buyer A pays Rs. 215 and buyer B pays Rs. 430, the equations are:
\[
  2x + y = 215, \quad 4x + 2y = 430.
\]
The second equation is a multiple of the first, representing the same line. Thus, infinitely many solutions exist.

\section*{Connection with Geometry}

\begin{itemize}
  \item Each linear equation represents a line in 2D space.
  \item The number of solutions corresponds to the intersection of lines:
    \begin{itemize}
      \item Unique solution: Lines intersect at a single point.
      \item No solution: Lines are parallel and do not intersect.
      \item Infinitely many solutions: Lines overlap completely.
    \end{itemize}
\end{itemize}

\section*{Matrix Representation Examples}

\textbf{Example 1:}
The system:
\[
  8x + 8y + 4z = 1960, \quad 12x + 5y + 7z = 2215, \quad 3x + 2y + 5z = 1135
\]
is represented as:
\[
  A =
  \begin{bmatrix}
    8 & 8 & 4 \\
    12 & 5 & 7 \\
    3 & 2 & 5
  \end{bmatrix},
  \quad
  x =
  \begin{bmatrix}
    x \\
    y \\
    z
  \end{bmatrix},
  \quad
  b =
  \begin{bmatrix}
    1960 \\
    2215 \\
    1135
  \end{bmatrix}.
\]

\textbf{Example 2:}
The system:
\[
  3x + 2y + z = 6, \quad x - \frac{1}{2}y + \frac{2}{3}z = \frac{7}{6}, \quad 4x + 6y - 10z = 0
\]
is represented as:
\[
  A =
  \begin{bmatrix}
    3 & 2 & 1 \\
    1 & -\frac{1}{2} & \frac{2}{3} \\
    4 & 6 & -10
  \end{bmatrix},
  \quad
  x =
  \begin{bmatrix}
    x \\
    y \\
    z
  \end{bmatrix},
  \quad
  b =
  \begin{bmatrix}
    6 \\
    \frac{7}{6} \\
    0
  \end{bmatrix}.
\]

\section*{Conclusion}

We studied systems of linear equations, their representation using matrices, and the types of solutions they can have. A system can have no solution, one unique solution, or infinitely many solutions. Matrix notation simplifies working with these systems, particularly for solving or analyzing them.

\end{document}
