\documentclass{article}
\usepackage{amsmath}
\usepackage{amssymb}
\usepackage{geometry}

\geometry{margin=1in}

\title{Lecture Summary: Row Reduction and Determinants}
\author{}
\date{}

\begin{document}

\maketitle

\section*{Source: Lec25.pdf}

\section*{Key Points}

\begin{itemize}
  \item \textbf{Elementary Row Operations:}
    \begin{itemize}
      \item \textbf{Type 1: Row Interchange.} Swap two rows, denoted as $R_i \leftrightarrow R_j$.
      \item \textbf{Type 2: Scalar Multiplication.} Multiply a row by a non-zero scalar $t$, denoted as $tR_i$.
      \item \textbf{Type 3: Row Addition.} Add a multiple of one row to another, denoted as $R_i \leftarrow R_i + tR_j$.
    \end{itemize}

  \item \textbf{Row Reduction:}
    \begin{itemize}
      \item A systematic method to transform a matrix into \textbf{Row Echelon Form (REF)} or \textbf{Reduced Row Echelon Form (RREF)}.
      \item Steps:
        \begin{enumerate}
          \item Identify the leftmost non-zero column.
          \item Use row operations to create a leading $1$ in the top position of this column.
          \item Use row operations to create zeros below this leading $1$.
          \item Repeat for the submatrix below the current row.
        \end{enumerate}
    \end{itemize}

  \item \textbf{Reduced Row Echelon Form:}
    \begin{itemize}
      \item In addition to REF:
        \begin{enumerate}
          \item Each leading $1$ in a row has zeros above and below it.
          \item Leading $1$s move from left to right in subsequent rows.
        \end{enumerate}
    \end{itemize}

  \item \textbf{Applications of Row Reduction:}
    \begin{itemize}
      \item Solve systems of linear equations $Ax = b$.
      \item Compute the determinant of square matrices.
    \end{itemize}

  \item \textbf{Determinants via Row Reduction:}
    \begin{itemize}
      \item Row reduce a square matrix to REF (upper triangular form).
      \item If the diagonal entries are all zeros, $\det(A) = 0$.
      \item If non-zero, compute $\det(A)$ as:
        \[
          \det(A) = \prod \text{(Diagonal Entries in REF)} \cdot \text{Product of scaling factors from Type 2 operations}.
        \]
      \item Type 1 row operations change the sign of the determinant, while Type 3 row operations do not affect it.
    \end{itemize}
\end{itemize}

\section*{Simplified Explanation}

\textbf{Example: Row Reduction to REF}
Given:
\[
  A =
  \begin{bmatrix}
    2 & 4 & 1 \\
    3 & 8 & 7 \\
    5 & 6 & 9
  \end{bmatrix}.
\]
Steps:
\begin{enumerate}
  \item Divide $R_1$ by $2$:
    \[
      \begin{bmatrix}
        1 & 2 & 0.5 \\
        3 & 8 & 7 \\
        5 & 6 & 9
      \end{bmatrix}.
    \]
  \item Subtract $3R_1$ from $R_2$ and $5R_1$ from $R_3$:
    \[
      \begin{bmatrix}
        1 & 2 & 0.5 \\
        0 & 2 & 5.5 \\
        0 & -4 & 6.5
      \end{bmatrix}.
    \]
  \item Divide $R_2$ by $2$ and eliminate the entry below $R_2(2,2)$ using $R_3 + 2R_2$:
    \[
      \begin{bmatrix}
        1 & 2 & 0.5 \\
        0 & 1 & 2.75 \\
        0 & 0 & 8
      \end{bmatrix}.
    \]
  \item Divide $R_3$ by $8$:
    \[
      \begin{bmatrix}
        1 & 2 & 0.5 \\
        0 & 1 & 2.75 \\
        0 & 0 & 1
      \end{bmatrix}.
    \]
\end{enumerate}

\textbf{Determinant Calculation}
\begin{itemize}
  \item Diagonal entries: $1, 1, 1$.
  \item Scaling factors from Type 2 operations: $\frac{1}{2} \cdot 2 \cdot 8 = 8$.
  \item Final determinant:
    \[
      \det(A) = 1 \cdot 1 \cdot 1 \cdot 8 = 8.
    \]
\end{itemize}

\section*{Conclusion}

In this lecture, we:
\begin{itemize}
  \item Defined row reduction and the elementary row operations.
  \item Showed how to transform matrices into REF and RREF systematically.
  \item Illustrated the computation of determinants using row reduction as an efficient alternative to the definition.
\end{itemize}

Row reduction is a versatile tool for solving linear systems and analyzing matrix properties. It simplifies computations, especially for large matrices.

\end{document}
