\documentclass{article}
\usepackage{amsmath}
\usepackage{amssymb}
\usepackage{geometry}

\geometry{margin=1in}

\title{Lecture Summary: Lengths and Angles in Vector Spaces}
\author{}
\date{}

\begin{document}

\maketitle

\section*{Source: Lec46.pdf}

\section*{Key Points}

\begin{itemize}
  \item \textbf{Dot Product in $\mathbb{R}^n$:}
    \begin{itemize}
      \item The dot product of two vectors in $\mathbb{R}^n$, $u = (u_1, u_2, \dots, u_n)$ and $v = (v_1, v_2, \dots, v_n)$, is defined as:
        \[
          u \cdot v = \sum_{i=1}^n u_i v_i.
        \]
      \item Example in $\mathbb{R}^2$: For $u = (3, 4)$ and $v = (2, 7)$:
        \[
          u \cdot v = 3 \cdot 2 + 4 \cdot 7 = 6 + 28 = 34.
        \]
    \end{itemize}

  \item \textbf{Length (Norm) of a Vector:}
    \begin{itemize}
      \item The length (or norm) of a vector $v = (v_1, v_2, \dots, v_n)$ is:
        \[
          \|v\| = \sqrt{v \cdot v} = \sqrt{\sum_{i=1}^n v_i^2}.
        \]
      \item Example in $\mathbb{R}^2$: For $v = (3, 4)$:
        \[
          \|v\| = \sqrt{3^2 + 4^2} = \sqrt{9 + 16} = 5.
        \]
      \item Example in $\mathbb{R}^3$: For $v = (4, 3, 3)$:
        \[
          \|v\| = \sqrt{4^2 + 3^2 + 3^2} = \sqrt{16 + 9 + 9} = \sqrt{34}.
        \]
    \end{itemize}

  \item \textbf{Angle Between Two Vectors:}
    \begin{itemize}
      \item The angle $\theta$ between two vectors $u$ and $v$ in $\mathbb{R}^n$ is given by:
        \[
          \cos(\theta) = \frac{u \cdot v}{\|u\| \|v\|}.
        \]
      \item $\theta$ can be calculated as:
        \[
          \theta = \cos^{-1}\left(\frac{u \cdot v}{\|u\| \|v\|}\right).
        \]
      \item Example in $\mathbb{R}^3$: For $u = (1, 0, 0)$ and $v = (1, 0, 1)$:
        \[
          u \cdot v = 1, \quad \|u\| = 1, \quad \|v\| = \sqrt{2}.
        \]
        \[
          \cos(\theta) = \frac{1}{1 \cdot \sqrt{2}} = \frac{1}{\sqrt{2}} \implies \theta = \frac{\pi}{4}.
        \]
    \end{itemize}

  \item \textbf{Key Observations:}
    \begin{itemize}
      \item The length of a vector $v$ is the square root of the dot product of $v$ with itself:
        \[
          \|v\| = \sqrt{v \cdot v}.
        \]
      \item The dot product and angle are related through cosine, providing a geometric interpretation of their relationship.
    \end{itemize}
\end{itemize}

\section*{Simplified Explanation}

\textbf{Dot Product}
The dot product of two vectors is a scalar obtained by multiplying their corresponding components and summing them:
\[
  u = (3, 4), \, v = (2, 7) \implies u \cdot v = 3 \cdot 2 + 4 \cdot 7 = 34.
\]

\textbf{Length of a Vector}
The length (norm) of a vector is computed as the square root of the sum of the squares of its components:
\[
  v = (3, 4) \implies \|v\| = \sqrt{3^2 + 4^2} = 5.
\]

\textbf{Angle Between Vectors}
The angle between two vectors is computed using the cosine formula:
\[
  \cos(\theta) = \frac{u \cdot v}{\|u\| \|v\|}.
\]
For $u = (1, 0, 0)$ and $v = (1, 0, 1)$:
\[
  \cos(\theta) = \frac{1}{\sqrt{2}} \implies \theta = \frac{\pi}{4}.
\]

\section*{Conclusion}

In this lecture, we:
\begin{itemize}
  \item Defined the dot product, length, and angle between vectors in $\mathbb{R}^n$.
  \item Demonstrated the geometric significance of these operations.
  \item Provided examples to compute lengths and angles in various dimensions.
\end{itemize}

These concepts are foundational in linear algebra, bridging algebraic and geometric interpretations of vector spaces.

\end{document}
