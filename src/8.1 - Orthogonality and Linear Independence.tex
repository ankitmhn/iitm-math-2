\documentclass{article}
\usepackage{amsmath}
\usepackage{amssymb}
\usepackage{geometry}

\geometry{margin=1in}

\title{Lecture Summary: Orthogonality and Linear Independence}
\author{}
\date{}

\begin{document}

\maketitle

\section*{Source: Lec48.pdf}

\section*{Key Points}

\begin{itemize}
  \item \textbf{Orthogonality in $\mathbb{R}^n$:}
    \begin{itemize}
      \item Two vectors $u$ and $v$ in $\mathbb{R}^n$ are orthogonal if the angle $\theta$ between them is $90^\circ$, which implies:
        \[
          \cos(\theta) = 0 \implies u \cdot v = 0.
        \]
      \item Example: The vectors $(1, 2, 3)$ and $(2, 2, -2)$ are orthogonal because:
        \[
          1 \cdot 2 + 2 \cdot 2 + 3 \cdot (-2) = 2 + 4 - 6 = 0.
        \]
    \end{itemize}

  \item \textbf{General Definition of Orthogonality:}
    \begin{itemize}
      \item In an inner product space $V$, two vectors $u$ and $v$ are orthogonal if:
        \[
          \langle u, v \rangle = 0,
        \]
        where $\langle \cdot, \cdot \rangle$ is the inner product.
      \item Example in $\mathbb{R}^2$ with a non-standard inner product:
        \[
          \langle u, v \rangle = x_1 y_1 - (x_1 y_2 + x_2 y_1) + 2x_2 y_2.
        \]
        Vectors $(1, 1)$ and $(1, 0)$ are orthogonal:
        \[
          \langle (1, 1), (1, 0) \rangle = 1 \cdot 1 - (1 \cdot 0 + 1 \cdot 1) + 2 \cdot 1 \cdot 0 = 0.
        \]
      \item Orthogonality depends on the chosen inner product.
    \end{itemize}

  \item \textbf{Orthogonal Sets:}
    \begin{itemize}
      \item A set of vectors $\{v_1, v_2, \dots, v_k\}$ in $V$ is orthogonal if:
        \[
          \langle v_i, v_j \rangle = 0 \quad \text{for all } i \neq j.
        \]
      \item Example: In $\mathbb{R}^3$ with the dot product, the set $\{(4, 3, -2), (-3, 2, -3), (-5, 18, 17)\}$ is orthogonal:
        \begin{align*}
          (4, 3, -2) \cdot (-3, 2, -3) &= -12 + 6 + 6 = 0, \\
          (4, 3, -2) \cdot (-5, 18, 17) &= -20 + 54 - 34 = 0, \\
          (-3, 2, -3) \cdot (-5, 18, 17) &= 15 + 36 - 51 = 0.
        \end{align*}
    \end{itemize}

  \item \textbf{Orthogonality and Linear Independence:}
    \begin{itemize}
      \item An orthogonal set of non-zero vectors is always linearly independent.
      \item Proof sketch:
        \begin{itemize}
          \item Assume $\sum_{i=1}^k c_i v_i = 0$.
          \item Taking the inner product with $v_1$, only the term $c_1 \langle v_1, v_1 \rangle$ remains (since $\langle v_i, v_1 \rangle = 0$ for $i \neq 1$).
          \item $\langle v_1, v_1 \rangle > 0 \implies c_1 = 0$.
          \item Repeating for all $v_i$, we conclude $c_i = 0$ for all $i$.
        \end{itemize}
    \end{itemize}

  \item \textbf{Orthogonal Basis:}
    \begin{itemize}
      \item A basis $\{v_1, v_2, \dots, v_n\}$ of $V$ is orthogonal if it is an orthogonal set.
      \item A basis is orthogonal if and only if it is a maximal orthogonal set.
      \item Example:
        \begin{itemize}
          \item In $\mathbb{R}^3$, $\{(4, 3, -2), (-3, 2, -3), (-5, 18, 17)\}$ is an orthogonal basis since it is orthogonal and has size 3, the dimension of $\mathbb{R}^3$.
          \item In $\mathbb{R}^2$ with the non-standard inner product:
            \[
              \langle u, v \rangle = x_1 y_1 - (x_1 y_2 + x_2 y_1) + 2x_2 y_2,
            \]
            the set $\{(1, 1), (1, 0)\}$ forms an orthogonal basis.
        \end{itemize}
    \end{itemize}
\end{itemize}

\section*{Simplified Explanation}

\textbf{Orthogonal Vectors:}
Vectors are orthogonal if their inner product is 0. In $\mathbb{R}^n$, this corresponds to being at right angles.

\textbf{Orthogonal Sets:}
A set of vectors is orthogonal if every pair in the set is orthogonal. Orthogonal sets are always linearly independent.

\textbf{Orthogonal Basis:}
An orthogonal basis is an orthogonal set that spans the entire vector space. For example, the standard basis in $\mathbb{R}^n$ is an orthogonal basis with the dot product.

\section*{Conclusion}

In this lecture, we:
\begin{itemize}
  \item Defined orthogonality using inner products.
  \item Explored the relationship between orthogonal sets and linear independence.
  \item Introduced orthogonal bases and provided examples in different inner product spaces.
\end{itemize}

Orthogonality simplifies the study of vector spaces, making linear independence checks and computations more efficient.

\end{document}
