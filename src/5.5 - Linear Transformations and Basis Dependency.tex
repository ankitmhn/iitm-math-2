\documentclass{article}
\usepackage{amsmath}
\usepackage{amssymb}
\usepackage{geometry}

\geometry{margin=1in}

\title{Lecture Summary: Linear Transformations and Basis Dependency}
\author{}
\date{}

\begin{document}

\maketitle

\section*{Source: Lec40.pdf}

\section*{Key Points}

\begin{itemize}
  \item \textbf{What is a Linear Transformation?}
    \begin{itemize}
      \item A \textbf{linear transformation} is a generalization of a linear mapping for arbitrary vector spaces $V$ and $W$.
      \item A function $f: V \to W$ is a linear transformation if:
        \[
          f(u + v) = f(u) + f(v), \quad f(c \cdot u) = c \cdot f(u),
        \]
        for all $u, v \in V$ and scalars $c \in \mathbb{R}$.
      \item This definition is equivalent to the property of linearity.
    \end{itemize}

  \item \textbf{Examples of Linear Transformations:}
    \begin{itemize}
      \item Linear mappings such as cost functions and matrix-vector multiplications are examples of linear transformations.
      \item Projections, rotations, and scaling in vector spaces are also linear transformations.
    \end{itemize}

  \item \textbf{One-to-One (Injective) and Onto (Surjective) Transformations:}
    \begin{itemize}
      \item A transformation $f: V \to W$ is \textbf{injective} if:
        \[
          f(v_1) = f(v_2) \implies v_1 = v_2.
        \]
      \item A transformation $f: V \to W$ is \textbf{surjective} if:
        \[
          \forall w \in W, \, \exists v \in V \, \text{such that } f(v) = w.
        \]
      \item Injectivity ensures no two distinct vectors in $V$ map to the same vector in $W$, while surjectivity ensures the entire range of $W$ is covered.
    \end{itemize}

  \item \textbf{Isomorphisms:}
    \begin{itemize}
      \item A linear transformation is an \textbf{isomorphism} if it is both injective and surjective.
      \item Example: The identity mapping $f(x) = x$ for $x \in \mathbb{R}^n$ is an isomorphism.
      \item Example of non-isomorphism:
        \[
          f(x, y) = (2x, 0) \quad \text{is injective but not surjective.}
        \]
    \end{itemize}

  \item \textbf{Role of Basis in Linear Transformations:}
    \begin{itemize}
      \item A basis for $V$ uniquely determines the action of a linear transformation $f: V \to W$.
      \item If $\{v_1, v_2, \dots, v_n\}$ is a basis for $V$, then $f$ is fully determined by the values $f(v_1), f(v_2), \dots, f(v_n)$.
      \item Changing the basis of $V$ results in a different linear transformation.
    \end{itemize}

  \item \textbf{Matrix Representation of Linear Transformations:}
    \begin{itemize}
      \item A linear transformation can be represented as a matrix $A$:
        \[
          f(x) = A \cdot x,
        \]
        where $x$ is the input vector and $A$ encodes the transformation.
      \item Different bases lead to different matrix representations.
    \end{itemize}
\end{itemize}

\section*{Simplified Explanation}

\textbf{Example 1: Linear Transformation Defined by Basis Values}
Let $f: \mathbb{R}^2 \to \mathbb{R}^2$ be defined by:
\[
  f(1, 0) = (2, 0), \quad f(0, 1) = (0, 1).
\]
\begin{itemize}
  \item For any $(x, y) \in \mathbb{R}^2$, write it as:
    \[
      (x, y) = x(1, 0) + y(0, 1).
    \]
  \item The transformation is:
    \[
      f(x, y) = x \cdot f(1, 0) + y \cdot f(0, 1) = (2x, y).
    \]
\end{itemize}

\textbf{Example 2: Changing the Basis}
Using basis $\{(1, 0), (1, 1)\}$:
\begin{itemize}
  \item Represent $(x, y)$ as:
    \[
      (x, y) = a(1, 0) + b(1, 1),
    \]
    where $a = x - y, b = y$.
  \item Transformation with new basis:
    \[
      f(x, y) = (2a, b) = (2(x - y), y).
    \]
\end{itemize}

\section*{Conclusion}

In this lecture, we:
\begin{itemize}
  \item Defined linear transformations and explained their relationship with linear mappings.
  \item Discussed injectivity, surjectivity, and isomorphisms.
  \item Highlighted the role of basis in determining linear transformations and how changes in basis alter transformations.
  \item Illustrated matrix representations of linear transformations and their basis dependency.
\end{itemize}

Linear transformations provide a framework for understanding linear relationships in any vector space, and basis choice is crucial for their representation and computation.

\end{document}
